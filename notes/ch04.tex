\chapter{Parabolic PDEs}

Heat equation:

\begin{equation*}
\frac{\partial u}{\partial t} (x, t) = c \Delta u(x,t) + f(x, t), \quad (x, t) \in \Omega \times [0, T]
\end{equation*}

where $\Omega$ is an open subject of $\mathbb{R}^d \; (d = 1, 2, \text{ or } 3)$ (interprets a body), $[0, T]$, in a time interval.
$u$ represents temperature; $c$ depends on conductivity, specific heat and density of the body $\Omega$; $f$ represents sources or sinks of heat.

We have to add:
\begin{itemize}
	\item boundary conditions: $u(x, t)=0, \quad (x,t) \in \Gamma \times [0, T]$ where $T$ is the border of $\Omega$;
	\item initial condition: $u(x, 0) = u_0(x), \quad x \in \Omega$
\end{itemize}

The problem: given $c, f, u_0$, find $u$!

\section{Numerical solution of the heat equation}

Consider $\Omega = (0, 1)^2$, $h=\frac{1}{N}$, with $N$ positive integer.
Then we have $\Omega_h, \Gamma_h, \Delta_h$.

The discrete problem is:

\begin{equation*}
\frac{\partial}{\partial t} u_h((x, y), t) = \Delta_h u_h(x, y) + f(x, y),
\quad (x, y) \in \Omega_h
\end{equation*}

\begin{equation*}
u_h((x, y), t) = 0, \quad (x, y) \in \Gamma_h
\end{equation*}

\begin{equation*}
u_h((x,y), 0) = u_0(x, y), \quad (x, y) \in \Omega_h
\end{equation*}

This is a linear system of $(N-1)^2$ scalar linear differential equations in the $(N-1)^2$ unknown functions:

\begin{equation*}
u_h ((x, y), \cdot): [0, T] \rightarrow \mathbb{R}, \quad (x, y) \in \Omega_h
\end{equation*}

This is only a \textit{semi-discretization} because the time is not discretized (it's called \textit{method of lines} since the lines are the unknown functions).
For a full discretization we need a numerical method for the time-integration.

The simplest method to solve ODEs of type
\begin{equation*}
y'(t) = f(t, y(t)), \quad t \in [0, T]
\end{equation*}
is the Explicit Euler (also called Forward Euler) method, which scheme is $y_{m+1} = y_m + k f(t_m, y_m), \quad m=0, \dots, M-1$.

There's also the Implicit Euler (also called Backward Euler) method: $y_{m+1} = y_m + f(t_{m+1}, y_{m-})$.

And also the Trapezoidal Rule: $y_{m+1} = y_m + \frac{1}{2} (f(t_m, y_m) + f(t_{m+1}, y_{m+1})$.

We consider the \textit{1D} case:

\begin{equation*}
\frac{\partial}{\partial t} u(x, t) = \frac{\partial^2}{\partial x^2} u(x, t) + f(x, t), \quad (x, t) \in \overbrace{(0, 1)}^{= \Omega} \times [0, T]
\end{equation*}

\begin{equation*}
u(0, t) = u(1, t) = 0
\end{equation*}
\begin{equation*}
u(x, 0) = u_0(x), \quad x \in (0, 1)
\end{equation*}

with: $\Omega = (0, 1)$, $\Omega_h = \{nh: n \in \{1, \dots, N-1\} \}$ and $\Gamma_h = \{0, 1\}$.

We discretize in space by the centered difference scheme for the second derivative in space and in time by the forward difference scheme for the first derivative in time. We use $h = \frac{1}{N}$ as space stepsize and $k = \frac{T}{M}$ as time stepsize.
discrete point in space: $nh$ with $n \in \{1, \dots, N-1\}$ 
discrete point in time: $mk$ with $k \in \{1, \dots, u\}$ 

$U_n^m$ denotes two discrete approximation of $u(nh, mk)$.
At the point $(nh, mk)$, $n \in \{1, \dots, N-1\}$ and $m \in \{0, 1, \dots, M-1\}$ we have the discrete equation:

\begin{equation*}
\underbrace{\frac{U_n^{m+1} - U_n^{m}}{k}}_{\approx \frac{\partial u}{\partial t}(x, t) \text{ with } x=nh,\,t=mk } = c \underbrace{\frac{U_{n-1}^{m} - 2U_n^{m} + U_{n+1}^{m}}{h^2}}_{\approx \frac{\partial u}{\partial x^2}(x, t)} + \underbrace{f_n^m}_{= f(nk, mk)}
\end{equation*}

\begin{equation*}
U_n^{m+1} = \frac{c k}{h^2} U_{n-1}^m + (1 - \frac{2 c k}{h^2}) U_n^m + \frac{c k}{h^2} U_{n+1}^m + k f_n^m =
\end{equation*}

\begin{equation*}
= U_n^m + \frac{ck}{h^2}U_{n-1}^m - \frac{2 c k}{h^2} U_n^m + \frac{c k}{h^2}U_{n+1}^m
\end{equation*}

Consider:
\begin{equation*}
U^m = (U_1^m, \dots, U_{N-1}^m), f^m = (f_1^m, \dots, f_{N-1}^m)
\end{equation*}

\begin{equation*}
U^{m+1} = U^m + k(\Delta_h U^m + f^m)
\end{equation*}

\begin{equation*}
\end{equation*}

\begin{equation*}
\end{equation*}

\begin{equation*}
\end{equation*}


\begin{equation*}
\end{equation*}

\begin{equation*}
\end{equation*}

\begin{equation*}
\end{equation*}

\begin{equation*}
\end{equation*}

\begin{equation*}
\end{equation*}

\begin{equation*}
\end{equation*}

\begin{equation*}
\end{equation*}

\begin{equation*}
\end{equation*}

\begin{equation*}
\end{equation*}

\begin{equation*}
\end{equation*}

\begin{equation*}
\end{equation*}

\begin{equation*}
\end{equation*}

\begin{equation*}
\end{equation*}

\begin{equation*}
\end{equation*}

\begin{equation*}
\end{equation*}

\begin{equation*}
\end{equation*}

\begin{equation*}
\end{equation*}

\begin{equation*}
\end{equation*}

\begin{equation*}
\end{equation*}

\begin{equation*}
\end{equation*}

\begin{equation*}
\end{equation*}

\begin{equation*}
\end{equation*}

\begin{equation*}
\end{equation*}

\begin{equation*}
\end{equation*}

\begin{equation*}
\end{equation*}

\begin{equation*}
\end{equation*}

\begin{equation*}
\end{equation*}

\begin{equation*}
\end{equation*}

\begin{equation*}
\end{equation*}

\begin{equation*}
\end{equation*}

\begin{equation*}
\end{equation*}

\begin{equation*}
\end{equation*}

\begin{equation*}
\end{equation*}

\begin{equation*}
\end{equation*}

\begin{equation*}
\end{equation*}

\begin{equation*}
\end{equation*}

\begin{equation*}
\end{equation*}

\begin{equation*}
\end{equation*}

\begin{equation*}
\end{equation*}

\begin{equation*}
\end{equation*}

\begin{equation*}
\end{equation*}

\begin{equation*}
\end{equation*}

\begin{equation*}
\end{equation*}

\begin{equation*}
\end{equation*}

\begin{equation*}
\end{equation*}

\begin{equation*}
\end{equation*}

\begin{equation*}
\end{equation*}

\begin{equation*}
\end{equation*}

\begin{equation*}
\end{equation*}

\begin{equation*}
\end{equation*}

\begin{equation*}
\end{equation*}

\begin{equation*}
\end{equation*}

\begin{equation*}
\end{equation*}

\begin{equation*}
\end{equation*}

\begin{equation*}
\end{equation*}

\begin{equation*}
\end{equation*}

\begin{equation*}
\end{equation*}

\begin{equation*}
\end{equation*}

\begin{equation*}
\end{equation*}

\begin{equation*}
\end{equation*}

\begin{equation*}
\end{equation*}

\begin{equation*}
\end{equation*}

\begin{equation*}
\end{equation*}

\begin{equation*}
\end{equation*}

\begin{equation*}
\end{equation*}
