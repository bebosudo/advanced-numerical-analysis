\chapter{Numerical methods for PDE}

PDE stands for Partial Differential Equations, and we are going to study in particular Finite Difference Methods, not Finite Element Methods.

PDEs can be divided in three categories: elliptic, parabolic and high-parabolic. We will see a test problem on each of these categories.

\section{Poisson 2D problem with Dirichlet boundary conditions}
\begin{equation}
\text{Poisson equation:\quad} \Delta u(x, y) = f(x, y), \quad (x, y) \in \Omega
\end{equation}

\begin{equation}
\text{Dirichlet Boundary condition:\quad} u(x, y) = g(x, y), \quad (x, y) \in \Gamma
\end{equation}

where $\Omega = (0, 1)^2$

% im:I

Example: the electric potential inside a domain with a defined potential inside, $f$, and a fixed potential, $g$, at the border.

The Poisson 2D problem can be solved analytically using \textit{separation of variables} by searching for solutions of this type:

\begin{equation}
	u(x, y) = X(x) Y(y)
\end{equation}

Discretization: fix N positive integer and put $h = \frac{1}{N}$ as the stepsize.

Now we discretize the plane $\mathbb{R}^2$ by the mesh:

\begin{equation}
	\mathbb{R}^2_h = \{(mh, nh): m,n \in \mathbb{Z}\}
\end{equation}

% im:2

For a mesh point $(mh, nh)$, with $m, n \in \mathbb{Z}$ we can define four neighbors:

% im:3

The discrete version of $\Omega$ is


\begin{equation}
	\Omega_h := \Omega \cap \mathbb{R}^2_h
\end{equation}

% 4

while the discrete version of $\Gamma$ is the set of points $\Gamma_h$ in $\mathbb{R}^2_h$ which are not in $\Omega_h$, but have a neighbor in $\Omega_h$.

Notice how $\Gamma_h \neq \Gamma \cap \mathbb{R}^2_h$ because the four corners $\{(0,0), (0,1), (1,0), (1,1)\}$ are not included. 

We also define $\overline{\Omega_h} = \Omega_h \cup \Gamma_h$: this is the discrete version of $\overline{\Omega_h}  = \Omega \cup \Gamma$.\\


The discrete version of the Poisson problem with Dirichlet Boundary conditions is:


\begin{equation}
\text{discrete Poisson equation:\enspace} \Delta_h u_h(x, y) = f(x, y), (x, y) \in \Omega_h
\end{equation}
	
\begin{equation}
\text{discrete Dirichlet Boundary condition:\enspace} u_h(x,y) = g(x, y), (x, y) \in \Gamma_h
\end{equation}


An operator is a map/function, whose domains and codomains are set of functions.

$\Delta_h$ is a discrete Laplacian, i.e. and operator such that:
\begin{itemize}
	\item it associates (as a map) to a mesh function $\overline{\Omega_h} \rightarrow \mathbb{R}$ an interior mesh function $\Omega_h \rightarrow \mathbb{R}$;
	
	\item it approximates the Laplacian, in the sense that, for function $v: \overline{\Omega_h} \rightarrow \mathbb{R}$ sufficiently smooth, which means $C^2(\overline{\Omega_h})$:
		
		\begin{equation}
			\Delta_h v |_{\overline{\Omega_h}} \; (x,y) \approx \Delta v(x, y) \quad (x, y) \in \Omega_h
		\end{equation}

	where $|_{\overline{\Omega_h}}$ is the restriction of $v$ to $\overline{\Omega_h}  \subseteq \overline{\Omega}$.	
\end{itemize}

We are going to show how to construct a discrete Laplacian.\\


Consider a function $v$ of only one variable $t$. We discretize $v''(t)$ by:

\begin{equation}
\frac{v(t - h) - 2 v(t) + v(t + h)}{h^2}
\end{equation}
.

%5

\begin{equation}
\frac{\frac{v(t + h) - v(t)}{h} - \frac{v(t) - v(t - h)}{h}}{h} \approx\\
\frac{v'(t) - v'(t-h)}{h} \approx\\
v''(t-h) \approx v''(t)
\end{equation}

By Taylor approximations under the assumptions that v is $C^4$:

\begin{equation}
v(t-h) = v(t) - v'(t) h + \frac{1}{2} v''(t) h^2 - \frac{1}{6} v'''(t) h^3 + \frac{1}{24} v^{(4)}(\alpha_h) h^4
\label{t-h}
\end{equation}

where $\alpha_h \in [t-h, t]$;

\begin{equation}
v(t+h) = v(t) + v'(t) h + \frac{1}{2} v''(t) h^2 + \frac{1}{6} v'''(t) h^3 + \frac{1}{24} v^{(4)}(\beta_h) h^4
\label{t+h}
\end{equation}

where $\beta_h \in [t, t+h]$.

By summing \ref{t-h} and \ref{t+h} we obtain:
\begin{equation}
v(t-h) + v(t+h) = 2 v(t) + v''(t) h^2 + \frac{1}{24} (v^{(4)}(\alpha_h) + v^{(4)}(\beta_h)) h^4
\end{equation}

\begin{equation}
v(t-h) - 2 v(t) + v(t+h) = v''(t) h^2 + \frac{1}{24} (v^{(4)}(\alpha_h) + v^{(4)}(\beta_h)) h^4
\end{equation}
\begin{equation}
\frac{v(t-h) - 2 v(t) + v(t+h)}{h^2} = v''(t) + \frac{1}{24} (v^{(4)}(\alpha_h) + v^{(4)}(\beta_h)) h^2
\end{equation}

So we got an approximation of the second derivative of the function $v$.

\subsection{Discretization of the Laplacian}

For a smooth function $v: \overline{\Omega_h} \rightarrow \mathbb{R}$ we approximate the Laplacian:\\


% 6

by:

\begin{equation}
\Delta_h v(x, y) = \frac{v(x-h, y) - 2v(x,y) + v(x+h, y)}{h^2} + \frac{v(x, y-h) - 2v(x,y) + v(x, y+h)}{h^2}
\end{equation}

at $(x, y) \in \Omega_h$.

% 7

The computation of $\Delta_h v(x, y)$ requires $v(x-h, y)$ and $v(x, y+h)$ with $(x-h, y), (x, y+h) \in \Gamma_h$.

This discretization of the Laplacian is called the \textit{Five-points discretization}:

% 8


\subsection{Assess approximation of the discrete Laplacian}

Consider $v: \overline{\Omega} \rightarrow \mathbb{R}$ of class $C^4$. For $(x, y) \in \Omega_h$:

\begin{equation}
|\Delta_h v {|}_{\overline{\Omega_h}} (x,y) \Delta v(x,y)|
\end{equation}

\begin{equation}
|\frac{v(x-h, y)  - 2v(x, y) + v(x+h, y)}{h} + \frac{v(x, y-h) - 2v(x, y) + v(x, y+h)}{h} - \frac{\partial^2 v}{\partial x^2}(x, y) - \frac{\partial^2 v}{\partial y^2}(x, y)|
\end{equation}

which is $\leq$ than: 

\begin{equation}
|\frac{v(x-h, y) - 2v(x, y) + v(x+h, y)}{h^2} - \frac{\partial^2 v}{\partial x^2}(x, y)| + |\frac{v(x, y-h) - 2v(x, y) + v(x, y+h)}{h^2} - \frac{\partial^2 v}{\partial y^2}(x, y)|
\label{buh}
\end{equation}
.

\ref{buh} is equal to:

\begin{equation}
|\frac{1}{24} ( \frac{\partial^4 v}{\partial x^4}(\alpha_h, y) + \frac{\partial^4 v}{\partial x^4}(\beta_h, y) \; h^2) | + |\frac{1}{24} ( \frac{\partial^4 v}{\partial y^4}(x, \gamma_h) + \frac{\partial^4 v}{\partial y^4}(x, \delta_h) \; h^2) |
\label{buh2}
\end{equation}

where $\alpha_h \in [x-h, x], \beta_h \in [x, x+h], \gamma_h \in [y-h, y], \delta_h \in [y, y+h]$.

\ref{buh2} (the error) can be bounded ($\leq$) by:

\begin{equation}
\frac{1}{6} \max\{ \max_{(x, y) \in \overline{\Omega}} |\frac{\partial^4 v}{\partial x^4} (x, y)|, \max_{(x, y) \in \overline{\Omega}} |\frac{\partial^4 v}{\partial y^4} (x, y)| \} \; h^2
\end{equation}

So we can conclude that:

\begin{equation}
\max_{(x,y) \in \Omega_h}  |\Delta_h v {|}_{\overline{\Omega_h}} (x,y) \Delta v(x,y)| \leq \frac{1}{6} \max\{ \max_{(x, y) \in \overline{\Omega}} |\frac{\partial^4 v}{\partial x^4} (x, y)|, \max_{(x, y) \in \overline{\Omega}} |\frac{\partial^4 v}{\partial y^4} (x, y)| \} \; h^2
\end{equation}


This is called the \textit{consistency error} of the five point approximation of the discrete Laplacian.

\subsection{Exercise}

Given $v \in C^6(\Omega)$, find  a function $C(x,y)$ of the point $(x, y) \in \overline{\Omega}$ and $M \ge 0$ such that 
\begin{equation*}
\max_{(x,y) \in \Omega_h}  |\Delta_h v {|}_{\overline{\Omega_h}} (x,y) - \Delta v(x,y) - C(x, y) h^2| \le M h^4
\end{equation*}

\begin{equation*}
%\max_{(x,y) \in \Omega_h}  |
\Delta_h v {|}_{\overline{\Omega_h}} (x,y) = \Delta v'(x,y) + \text{error}
\end{equation*}

with $\text{error} = O(h^2)(*) = C(x, y) h^2 + O(h^4) (**)$.


Remind that:
\begin{equation}
v''(t) \approx \frac{v(t - h) - 2 v(t) + v(t + h)}{h^2}
\end{equation}

we can continue with the Taylor expansion:
\begin{equation*}
v(t-h) = v(t) - v'(t) h + \frac{v''(t)}{2} h^2 - \frac{v'''(t)}{6} h^3 + \frac{v^{4}(t)}{24} h^4  - \frac{v^{5}(t)}{120} h^5 + \frac{v^{6}(t)}{6!} h^6
\end{equation*}

\begin{equation*}
v(t-h) = v(t) + v'(t) h + \frac{v''(t)}{2} h^2 + \frac{v'''(t)}{6} h^3 + \frac{v^{4}(t)}{24} h^4  + \frac{v^{5}(t)}{120} h^5 + \frac{v^{6}(t)}{6!} h^6
\end{equation*}

where $\alpha \in [t-h, t]$ and $\beta \in [t, t+h]$.

\begin{equation*}
v(t-h) + v(t+h) = 2v(t) + v''(t) h^2 + \frac{v^{4}(t)}{12} h^4 + \frac{v^{6}(\alpha)}{6!} h^6 + \frac{v^{6}(\beta)}{6!} h^6
\end{equation*}

\begin{equation*}
\frac{v(t-h) - 2v(t)+ v(t+h)}{h^2} - v''(t) h^2 = \frac{v^{4}(t)}{12} h^4 + \frac{v^{6}(\alpha)}{6!} h^6 + \frac{v^{6}(\beta)}{6!} h^6
\end{equation*}


\dots


The discrete problem:
\begin{equation*}
\text{Poisson equation:\quad} \Delta u_h(x, y) = f(x, y), (x, y) \in \Omega_h
\end{equation*}

\begin{equation*}
\text{Dirichlet Boundary condition:\quad} u_h(x, y) = g(x, y), (x, y) \in \Gamma_h
\end{equation*}

is a linear system of $(N-1)^2$ unknowns (where $N$ is the number of subdivisions we do in each dimension of the square)  and $(N-1)^2$ equations.

The unknowns are $u_{ij} := u_h (ih, jh)$ with $i \in \{1, \dots, N-1 \}$ and $j \in \{1, \dots, N-1 \}$.

% 1 : 2018/03/15

.. we don't consider the values on the borders as unknowns because they are known, from the second equation of the Dirichlet Boundary condition.

On the border $\Gamma_h$:

\begin{equation*}
u_{ij} = u_h (ih, jh) = h_{ij} := g(ih, jh)
\end{equation*}

where 
\begin{equation*}
(i,j) \in \{0\} \times \{1, \dots, N-1 \} \cup \{1, \dots, N-1 \} \times \{0\} \cup \{N\} \times  \{1, \dots, N-1 \} \cup  \{1, \dots, N-1 \} \times \{N\}
\end{equation*}

The equations are 
\begin{equation*}
\Delta_h u_h (x, y) = f(x, y), \quad \in \Omega_h
\end{equation*}

each equation corresponds to a point in $\Omega_h$ with $(N-1)^2$ equations.

The equation corresponding to the point $(ih, jh)$, with $i \in  \{1, \dots, N-1 \}, j \in  \{1, \dots, N-1 \}$ is:

\begin{equation*}
\frac{u_{i-1\; j} - 2 u_{ij} + u_{i+1\; j}}{h^2} + \frac{u_{i\; j-1} - 2 u_{ij} + u_{i\; j+1}}{h^2} = f(ih, jh) = f_{ij}
\end{equation*}

\begin{equation*}
u_{i\; j-1} + u_{i-1\; j} -4 u_{ij} +  u_{i+1\; j} + u_{i\; j+1} = h^2 f_{ij}
\end{equation*}

In order to apply a linear system of the form $A u = b$, we need to re-arrange the equations of the points of the square (which are 2D) into a 1D vector.

Consider $N=4$:

% 2
\dots

We want to rewrite our matrix of unknowns $u$:

\begin{equation*}
u = \begin{bmatrix}
u_{11} & u_{21} & u_{31} \\
u_{12} & u_{22} & u_{32} \\
u_{13} & u_{23} & u_{33} \\
\end{bmatrix}
\end{equation*}

as a vector:

\begin{equation*}
u = \begin{bmatrix}
u_{11} \\
u_{21} \\
u_{31} \\
u_{12} \\
u_{22} \\
u_{32} \\
u_{13} \\
u_{23} \\
u_{33} \\
\end{bmatrix}
\end{equation*}

Then we can rewrite our equation $u_{i\; j-1} + u_{i-1\; j} -4 u_{ij} +  u_{i+1\; j} + u_{i\; j+1} = h^2 f_{ij}$ as (for the point $(1,1)$):

\begin{equation*}
u_{10} + u_{01} - 4u_{11} + u_{21}  + u_{12} = h^2 f_{11}
\end{equation*}

where $u_{10}$ and $u_{01}$ correspond to $g_{10}$ and $g_{01}$ respectively.

\begin{equation*}
- 4u_{11} + u_{21}  + u_{12} = h^2 f_{11} - u_{10} - u_{01}
\end{equation*}

For the point $(2,1)$:

\begin{equation*}
u_{20} + u_{11} - 4u_{21} + u_{31}  + u_{22} = h^2 f_{22}
\end{equation*}

rewritten as:
\begin{equation*}
u_{11} - 4u_{21} + u_{31} + u_{22} = h^2 f_{22} - g_{20}
\end{equation*}

We can write a matrix $\mathcal{A}$:

\begin{equation*}
\mathcal{A} = \begin{bmatrix}
-4 &  1 &  0 &  1 &  0 &  0 &  0 &  0 &  0 \\
 1 & -4 &  1 &  0 &  1 &  0 &  0 &  0 &  0 \\
 0 &  1 & -4 &  0 &  0 &  1 &  0 &  0 &  0 \\
 1 &  0 &  0 & -4 &  1 &  0 &  1 &  0 &  0 \\
 0 &  1 &  0 &  1 & -4 &  1 &  0 &  1 &  0 \\
 0 &  0 &  1 &  0 &  1 & -4 &  0 &  0 &  1 \\
 0 &  0 &  0 &  1 &  0 &  0 & -4 &  1 &  0 \\
 0 &  0 &  0 &  0 &  1 &  0 &  1 & -4 &  1 \\
 0 &  0 &  0 &  0 &  0 &  1 &  0 &  1 & -4 \\
\end{bmatrix}
\end{equation*}

and the unknown vector $u$:

\begin{equation*}
u = \begin{bmatrix}
h^2 f_{11} - g_{01} - g_{10} \\
h^2 f_{21} - g_{20}          \\
h^2 f_{31} - g_{14}          \\
h^2 f_{12} - g_{02}          \\
h^2 f_{22}                   \\
h^2 f_{32} - g_{24}          \\
h^2 f_{13} - g_{03} - g_{14} \\
h^2 f_{23} - g_{24}          \\
h^2 f_{33} - g_{34} - g_{43} \\
\end{bmatrix}
\end{equation*}

$\mathcal{A}$ can be rewritten as a $3x3$ blocks matrix with blocks of size $3x3$:
\begin{equation*}
\mathcal{A} = \begin{bmatrix}
  A & I_3 &   0 \\
I_3 &   A & I_3 \\
  0 & I_3 &   A \\
\end{bmatrix}
\end{equation*}


For a general $N$, $\mathcal{A}$ is a $(N - 1) \times (N - 1)$ block matrix:
\begin{equation*}
\mathcal{A} = \begin{bmatrix}
      A & I_{N-1} &        &         &         \\
I_{N-1} &       A & \ddots &         &         \\
        &  \ddots & \ddots &  \ddots &         \\
%
        &         & \ddots &       A & I_{N-1} \\
        &         &        & I_{N-1} &       A \\
\end{bmatrix}
\end{equation*}

% -4  1
%  1 -4


A generic matrix with dimension $n \times n$ is called ``sparse" if the number of non-zero elements is $O(n)$.

The matrix $\mathcal{A}$ is sparse, block \textit{tridiagonal} (it has elements on the diagonal, on the diagonal of the upper matrix and on the diagonal of the lower matrix), symmetric ($\mathcal{A}^T = \mathcal{A}$) and negative definite (all the eigenvalues are negative).

Consider a symmetric (real eigen-values) matrix B $n \times n$, B is negative definite when:

\begin{equation*}
<x, Bx> = x^T B x = \sum_{i=1}^{m} \sum_{j=1}^{n} b_{ij} x_i x_j
\end{equation*}

is $ \le 0 \;\; \forall x: n \times 1$ vectors (negative semi-definite) and\\
$< 0$ for $x \ne 0$.

\subsection{$\mathcal{A}$ is negative definite}
We are going to prove that $\mathcal{A}$ is negative definite, which then means that it is a non-singular matrix, which means that it can be inverted, which means that our system can be solved.

First we prove that:

\begin{equation*}
<v, \mathcal{A}v> \le -2 \; {||v||}^2
\end{equation*}
for all $v \in \mathbb{R}^{N-1}$.

\begin{equation*}
<v, \mathcal{A}v> = \sum_{i=1}^{N-1} \sum_{j=1}^{N-1} a_{ij} v_i v_j =
\end{equation*}
\begin{equation*}
-4 v_{1}^{2} + v_1 v_2 + v_2 v_1 -4 v_2^2 + v_2 v_3 + v_3 v_2 - 4 v_3^2 + \dots -4 v_{N-2}^2 + v_{N-2} v_{N-1} + v_{N-1} v_{N-2} -4 v_{N-1}^2 =
\end{equation*}
\begin{equation*}
-4 v_1^2 -4 v_2^2 -4 v_3^2 + \dots -4 v_{N-2}^2 - 4 v_{N-1}^2 + 2 v_1 v_2 + 2 v_2 v_3 + \dots + 2 v_{N-2} v_{N-1} =
\end{equation*}
\begin{equation*}
-3 v_1^2 - 2 v_2^2 - 2 v_3^2 + \dots - 2 v_{N-2}^2 - 3 v_{N-1}^2 - (v_1^2 - 2 v_1 v_2 + v_2^2) - (v_2^2 - 2 v_2 v_3 + v_3^2) + \dots + ( v_{N-2}^2 - 2 v_{N-2} v_{N-1} + v_{N-1}^2) =
\end{equation*}
\begin{equation*}
-3 v_1^2 - 2 v_2^2 - 2 v_3^2 + \dots - 2 v_{N-2}^2 - 3 v_{N-1}^2 - (v_1 - v_2)^2 - (v_2 - v_3)^2 + \dots - (v_{N-2} - v_{N-1})^2 \le
\end{equation*}
\begin{equation*}
-2 v_1^2 - 2 v_2^2 - 2 v_3^2 + \dots - 2 v_{N-2}^2 - 2 v_{N-1}^2 = -2 \underbrace{(v_1^2 + v_2^2 + v_3^2 + \dots + v_{N-2}^2 + v_{N-1}^2)} %{= || v ||_2^2}
\end{equation*}


Consider $v = (v_1, \dots, v_{N-1}) \in \mathbb{R}^{(N-1)^2}$, where $v_1, \dots, v_{N-1} \in \mathbb{R}^{N-1}$.

\begin{equation*}
<v, \mathcal{A}v> = v^T \mathcal{A} v = [v_1^T, v_2^T, \dots, v_{N-1}^T]
\end{equation*}


\begin{equation*}
<x, Bx> = \sum_{i=1}^{m} \sum_{j=1}^{n} b_{ij} x_i x_j = \sum_{i=1}^{m} \sum_{j=1}^{n} <v_i, \mathcal{A}_ij v_j> = 
\end{equation*}
\begin{equation*}
<v_1, A v_1> + <v_1, I v_2> + <v_2, I v_1> + <v_2, A v_2> + <v_2, I v_3> + <v_3, I v_2> + <v_3, A v_3> + \dots +
\end{equation*}
\begin{equation*}
+ <v_{N-2}, A v_{N-2}> + <v_{N-2}, v_{N-1}> + <v_{N-1}. v_{N-2}> + <v_{N-1}. A v_{N - 1}> =
\end{equation*}
\begin{equation*}
\overbrace{<v_1, A v_1>}^{\le -2 ||v_1||_2^2} + \overbrace{<v_1, I v_2>}^{\le -2 || v_2 ||_2^2} + \overbrace{<v_2, I v_1>}^{\le -2 || v_3 ||_2^2} + \dots + \overbrace{<v_{N-2}, A v_{N-2}>}^{\le -2 || v_{N-2} ||_2^2} + \overbrace{<v_{N-1}, A v_{N - 1}>}^{\le -2 || v_{N-1} ||_2^2} + 2 \dots\dots \le
\end{equation*}
\begin{equation*}
-2 ||v_1||_2^2 -2 ||v_2||_2^2 +\dots -2 ||v_{N-1}||_2^2 \dots\dots =
\end{equation*}
\begin{equation*}
= - ||v_1||_2^2 - (||v_1||_2^2 -2 <v_1, v_2> + ||v_2||_2^2) - (||v_2||_2^2 -2 <v_2, v_3> + ||v_3||_2^2) + \dots - (||v_{N-2}||_2^2 -2 <v_{N-2}, v_{N-1}> + ||v_{N-1}||_2^2) - ||v_{N-1}||_2^2 =
\end{equation*}
\begin{equation*}
= - ||v_1||_2^2 - ||v_1 - v_2||_2^2 - ||v_2 - v_3||_2^2 + \dots - ||v_{N-2} - v_{N-1}||_2^2 - ||v_{N-1}||_2^2 \le 0
\end{equation*}

Now we show that $<v, \mathcal{A}v> < 0$ for  $v \ne 0$.

Equivalently, we show that $<v, \mathcal{A}v> = 0 \implies v = 0$.
When $<v, \mathcal{A}v> = 0$, also 

\begin{equation*}
- ||v_1||_2^2 - ||v_1 - v_2||_2^2 - ||v_2 - v_3||_2^2 + \dots - ||v_{N-2} - v_{N-1}||_2^2 = 0
\end{equation*}

\dots

\begin{equation*}
||v_1||_2 = ||v_1 - v_2||_2 = \dots = ||v_{N-2} - v_{N-1}||_2 = ||v_{N-1}||_2 = 0
\end{equation*}
and so:
\begin{equation*}
v_1 = 0, v_1 = v_2, \dots, v_{N-1} = 0
\end{equation*}

and so:
\begin{equation*}
v_1 = v_2 = \dots = v_{N-1} = 0
\end{equation*}
and so:
\begin{equation*}
v = 0
\end{equation*}

\subsubsection*{Exercise}

Write the 1D Poisson problem with Dirichlet Boundary condition on $\Omega = (0, 1)$.

By definition, the Laplacian is: $\frac{\partial^2}{\partial x^2} (x,y) + \frac{\partial^2 u}{\partial y^2} (x,y)$.

The 1D version of the Poisson problem is: $u''(x) = f(x), \quad x \in \Omega = (0,1)$
$u(x) = g(x), \quad x \in \Gamma$ : $u(0) = g(0), u(1) = g(1)$.


Solve analytically this problem:

$u'(x) = u'(0) + \int_{0}^{x} \underbrace{u''(s)}_{f(s)} ds$, \quad $x \in [0, 1]$.

\begin{equation*}
u(x) = u(0) + \int_{0}^{x} u'(s) ds = u(0) + \int_{0}^{x} (u'(0) + (\int_{0}^{s} f(\delta) d\delta) \;\, ds = 
\end{equation*}
\begin{equation*}
u(0) + u'(0) \int_{0}^{x} ds + \int_{0}^{x} \int_{0}^{s} f(\delta) d\delta ds = 
\end{equation*}

\begin{equation*}
u(x) = u(0) + u'(0) x + \int_{0}^{x} \int_{0}^{s} f(\delta) d\delta ds \quad x \in [0, 1]
\end{equation*}

$u(0) = g(0)$
$u(1) = u(0) + u'(0) + \int_{0}^{1} \int_{0}^{s} f(\delta) d\delta ds \quad x \in [0, 1]$

\dots\dots


Propose a corresponding discrete problem and write the associated linear system:

$\Omega = (0, 1)$, $h = 1/N$ with $N$ positive integer.
$\Omega_h = {h, 2h, \dots, (N-1) h}$

The discrete problem is:

\begin{equation*}
\Delta_h u_h (x) = f(x), \quad x \in \Omega_h
u_h (0) = g(0)
u_h (1) = g(1)
\end{equation*}

$\Delta_h$ associates to a function $\overline{\Omega_h} \rightarrow \mathbb{R}$ a function $\Omega_h \rightarrow \mathbb{R}$ that approximates $\Delta u = u''$.

For a function $v_h: \overline{\Omega_h} \rightarrow \mathbb{R}$, we define:
\begin{equation*}
\Delta_h v_h(x) = \frac{v_h(x-h) - 2 v_h(x) + v_h(x+h)}{h^2}, \quad x \in \Omega_h
\end{equation*}

We can rewrite the discrete problem as:
\begin{equation*}
u_i = u_h(ih), \quad i \in {1, \dots, N-1}
\end{equation*}
\begin{equation*}
f_i = f(ih)
\end{equation*}
\begin{equation*}
g_0 = g(0), \; g_N = g(1)
\end{equation*}

\begin{equation*}
\frac{u_{i-1} - 2 u_i + u_i+1}{h^2} = f_i, \quad i \in {1, \dots, N-1}
\end{equation*}

with $u_0 = g_0, \quad u_N = g_N$,

\begin{equation*}
u_{i-1} - 2 u_i + u_i+1 = h^2 f_i, \quad i \in {1, \dots, N-1}
\end{equation*}

with $u_0 = g_0, \quad u_N = g_N$.

The linear system is:

We can write a matrix:

\begin{equation*}
\begin{bmatrix}
-2 &  1 &    &    &    &    &    \\
 1 & -2 &  1 &    &    &    &    \\
   &  1 & \ddots &  \ddots &    &    &    \\
 & & \ddots &   &   & & \\

   &    &    &    &  1 & -2 &  1 \\
   &    &    &    &    &  1 & -2 \\
\end{bmatrix}
\end{equation*}

and the unknown vector is:

\begin{equation*}
u = \begin{bmatrix}
h^2 f_{1} - g_{0}   \\
h^2 f_{2}           \\
h^2 f_{3}           \\
\dots               \\
h^2 f_{N-2}         \\
h^2 f_{N-1} - g_{N} \\
\end{bmatrix}
\end{equation*}

For $v \in \mathbb{R}^{N-1}$: $<v, \mathcal{A}v$ = $-2 v_1^2 + v_1 v_2 + v_2 v_1 - 2 v_2^2 + \dots$\\
$. \qquad \quad \qquad \qquad \qquad \qquad \qquad \qquad  + v_{N-2} v_{N-1} + v_{N-1} v_{N-2} + 2 v_N^2 =$

\begin{equation*}
- v_1^2 - (v_1 - v_2)^2 - (v_2 - v_3)^2 + \dots - (v_{N-2} - v_{N-1})^2 - v_{N-1}^2 \le 0
\end{equation*}

If $<v, \mathcal{A}v> = - v_1^2 - (v_1 - v_2)^2 + \dots - (v_{N-2} - v_{N-1})^2 - v_{N-1}^2 = 0$ then $v_1 = 0, v_1 - v_2 = 0, \dots, v_{N-2} - v_{N-1} = 0, v_{N-1} = 0$ and so $v = 0$.


\subsubsection*{Exercise}

Consider the 3D Poisson problem with Dirichlet boundary conditions on $\Omega = (0, 1)^3$.
Propose a discrete Laplacian and a consequent discrete problem.

\begin{equation*}
\Delta u(x, y, z) = f(x, y, z), \quad (x, y, z) \in \Omega = (0, 1)^3
\end{equation*}

\begin{equation*}
u(x, y, z) = g(x, y, z), \quad (x, y, z) \in \Gamma
\end{equation*}

\begin{equation*}
\Delta u(x, y, z) = \frac{\partial^2 u}{\partial x^2} (x, y, z) + \frac{\partial^2 u}{\partial y^2} (x, y, z) + \frac{\partial^2 u}{\partial z^2} (x, y, z)
\end{equation*}

\begin{equation*}
\Omega_h = \Omega \cap \mathbb{R}^3_h
\end{equation*}

\begin{equation*}
\mathbb{R}^3_h = {(mh, uh, ph) : m,n,p \in \mathbb{Z}}
\end{equation*}

$\Gamma_h$ in 2D was the border without the corners. In 3D is the set of faces, without the edges and the corners.
$\Gamma_h = $ "points of $\mathbb{R}^3_h \cap \Gamma$ with a neighbor in $\Omega_h$"

The discrete problem then is 

\begin{equation*}
\Delta_h u(x, y, z) = f(x, y, z), \quad (x, y, z) \in \Omega_h
\end{equation*}

\begin{equation*}
u_h(x, y, z) = g(x, y, z), \quad (x, y, z) \in \Gamma_h
\end{equation*}

The discrete Laplacian is given by a mesh function $v: \overline{\Omega_h} = \Omega_h \cup \Gamma_h \rightarrow \mathbb{R}$

\begin{equation*}
\Delta_h v_h(x, y, z) = \frac{v(x-h, y, z) - 2v(x,y,z) + v(x+h, y, z)}{h^2} +\\
\frac{v(x, y-h, z) - 2v(x,y,z) + v(x, y+h, z)}{h^2} + \frac{v(x, y, z-h) - 2v(x,y,z) + v(x, y, z+h)}{h^2} \quad (x, y, z) \in \Omega_h
\end{equation*}

Describe in the matrix of the linear system associated to the discrete problem the row corresponding to a mesh point in $\Omega_h$ with nearest neighbors in $\Omega_h$.

In order to write our problem in this way: $\mathcal{A} U = b$ we need to order the unknowns.
There are $(N-1)^3$ unknowns, where $h = \frac{1}{N}$; therefore $\mathcal{A}$ is a $(N-1)^3 \times (N-1)^3$ matrix.

What is the equation of a point whose neighbors are part of $\Omega_h$?

We construct the matrix from the equation $\Delta_h u_h(x, y, z) = f(x, y, z)$; which becomes $h^2 \Delta_h u_h(x, y, z) = h^2 f(x, y, z)$

%\begin{equation*}
$
-6 \overbrace{u(x, y, z)}^{= u_{ijk} = U_l} + \overbrace{u_h(x-h, y, z)}^{= u_{l-1}} + \overbrace{u_h(x+h, y, z)}^{= u_{l+1}} +\\
.\qquad\qquad\qquad \; \; + \overbrace{u_h(x, y-h, z)}^{= u_{l-(N-1)}} + \overbrace{u_h(x, y+h, z)}^{= u_{l+(N-1)}} +\\
.\qquad\qquad\qquad \; \; + \overbrace{u_h(x, y, z-h)}^{= u_{l- (N-1)^2}} + \overbrace{u_h(x, y, z+h)}^{= u_{l+ (N-1)^2}} =
h^2 f(x, y, z)
$
%\end{equation*}

\begin{equation*}
%U_{l-(N-1)^2} + U_{l-(N-1)^} + U_{l-1} - 6 U_{l} + U_{l+1} + U_{l+(N-1)} + U_{l+(N-1)^2} = h^2 f_l
\end{equation*}

\dots

\section{Convergence analysis of the discrete Poisson problem (with D. b.c)}

We assume that the continuous problem has a unique solution $u$.

We want to prove that the discrete problem has a unique solution $u_h$ (already done, but we obtain this by using another technique) and we want to give a bound on the error:

\begin{equation*}
\max_{(x,y) \in \overline{\Omega}_h} |u_h (x, y) - u(x, y)|
\end{equation*}

This bound will prove the convergence of $\mu_h$ to $u$ as $h \rightarrow 0$.


\subsection{Theorem: Discrete maximum principle}

Let $v_h: \overline{\Omega}_h \rightarrow \mathbb{R}$ such that:

\begin{equation*}
\Delta_h v_h \ge 0 \qquad \forall (x, y) \in \Omega_h
\end{equation*}

Then:
\begin{equation*}
\max_{(x,y) \in \Omega_h} v_h (x, y) \le \max_{(x,y) \in \Gamma_h} v_h (x, y)
\end{equation*}

with equality iff $v_h$ is constant.

\subsubsection{Continuous maximum principle}

Let $v: \overline{\Omega} \rightarrow \mathbb{R}$ of class $C^2$ such that:

\begin{equation*}
\Delta v (x,y) \ge 0 \qquad \forall (x, y) \in \Omega
\end{equation*}

Then:
\begin{equation*}
\sup_{(x,y) \in \Omega} v(x, y) \le \max_{(x,y) \in \Gamma} v(x, y)
\end{equation*}


\subsubsection{Proof of the Discrete maximum principle theorem}

We have to prove that:
\begin{equation*}
\max_{(x,y) \in \Omega_h} v_h(x, y) \le \max_{(x,y) \in \Gamma_h} v_h(x, y) 
\end{equation*}

with $=$ iff $v_h$ is constant.

We begin by observing if $v_h$ is constant, then:
\begin{equation*}
\max_{(x,y) \in \Omega_h} v_h(x, y) = \max_{(x,y) \in \Gamma_h} v_h(x, y) 
\end{equation*}

Given two logical sentences $A$ and $B$, $A \implies B$ is equivalent to $\neg B \implies \neg A$

When $v_h$ is not constant ($A$), then ($B$):
\begin{equation*}
\max_{(x,y) \in \Omega_h} v_h(x, y) < \max_{(x,y) \in \Gamma_h} v_h(x, y) 
\end{equation*}

Therefore we prove the counter-opposite ($\neg B$) of the previous equation.
\begin{equation*}
\max_{(x,y) \in \Omega_h} v_h(x, y) > \max_{(x,y) \in \Gamma_h} v_h(x, y) 
\end{equation*}
which means that $v_h$ is not constant ($\neg A$).

Suppose:
\begin{equation*}
\max_{(x,y) \in \Omega_h} v_h(x, y) \ge \max_{(x,y) \in \Gamma_h} v_h(x, y) 
\end{equation*}

So:
\begin{equation*}
\max_{(x,y) \in \Omega_h} v_h(x, y) = \max_{(x,y) \in \Gamma_h} v_h(x, y) 
\end{equation*}

Consider now a point $(x_0, y_0) \in \Omega_h$ where $v_h$ is the maximum.

We prove that all $(x_0 - h, y_0), (x_0 + h, y_0), (x_0, y_0 - h), (x_0, y_0 + h)$ are maximum points for $v_h$.

We have:
\begin{equation*}
h^2 \Delta_h v_h(x_0, y_0) \ge 0
\end{equation*}
\begin{equation*}
= v_h(x_0, y_0 - h) + v_h(x_0 - h, y_0) - 4 v_h(x_0, y_0) + v_h(x_0 + h, y_0) + v_h(x_0, y_0 + h)
\end{equation*}

and then:
\begin{equation*}
4 v_h(x_0, y_0) \le \overbrace{v_h(x_0, y_0 - h)}^{v_h (x_0, y_0)} + \overbrace{v_h(x_0 - h, y_0)}^{v_h (x_0, y_0)} + \overbrace{v_h(x_0 + h, y_0)}^{v_h (x_0, y_0)} + \overbrace{v_h(x_0, y_0 + h)}^{v_h (x_0, y_0)}
\end{equation*}
which means that the second member must be $\le 4 v_h (x_0, y_0)$.

Thus:
\begin{equation*}
4 v_h(x_0, y_0) = v_h(x_0, y_0 - h) + v_h(x_0 - h, y_0) + v_h(x_0 + h, y_0) + v_h(x_0, y_0 + h)
\end{equation*}

and it follows that:
\begin{equation*}
v_h(x_0, y_0 - h) = v_h(x_0 - h, y_0) = v_h(x_0 + h, y_0) = v_h(x_0, y_0 + h) = v_h(x_0, y_0)
\end{equation*}

We have proved that $(x_0 - h, y_0), (x_0 + h, y_0), (x_0, y_0 - h), (x_0, y_0 + h)$ are maximum points for $v_h$; this argument can be repeated recursively for all the neighbors, which means that they all have value $v_{max}$; we obtain that all the mesh points obtain the maximum value $v_{max}$, and $v_h$ is constant. $\blacksquare$

\subsection{Corollary of the Discrete maximum principle}

Let $v_h: \overline{\Omega}_h \rightarrow \mathbb{R}$ such that:

\begin{equation*}
\Delta_h v_h (x,y) \le 0 \qquad \forall (x, y) \in \Omega_h
\end{equation*}

\begin{equation*}
\min_{(x,y) \in \Omega_h} v_h(x, y) \ge \min_{(x,y) \in \Gamma_h} v_h(x, y) 
\end{equation*}


\subsubsection{Proof}
Consider $w_h = - v_h$. We have:

\begin{equation*}
\Delta_h w_h (x, y) = - \Delta_h v_h (x, y) \ge 0 \forall (x, y) \in \Omega_h
\end{equation*}

From the discrete maximum principle, we have:
\begin{equation*}
\max_{(x, y) \in \Omega_h} \overbrace{w_h (x, y)}^{= - v_h(x, y)} \le \max_{(x, y) \in \Gamma_h} \overbrace{w_h (x, y)}^{= - v_h(x, y)}
\end{equation*}
$\blacksquare$

\subsection{Theorem:}

The discrete Poisson problem (with D. b.c.) has a unique solution.

\subsubsection{Proof:}

The discrete problem is written as a linear system: $\mathcal{A} U = b$.

Existence and uniqueness of the solution of the discrete problem follows by the non singularity of the matrix $\mathcal{A}$.

$\mathcal{A}$ is non singular if:

\begin{equation*}
\mathcal{A} U = 0 \implies U = 0
\end{equation*}

We are now going to prove this implication.

$\mathcal{A} U = 0$ is obtained for the discrete problem:


\begin{equation*}
\Delta_h u_h(x, y) = 0, \quad (x, y) \in \Omega_h
\end{equation*}

\begin{equation*}
u_h(x, y) = 0, \quad (x, y) \in \Gamma_h
\end{equation*}

Now we show that $u_h = 0$ (i.e. $U = 0$).

We have:

\begin{equation*}
\Delta_h u_h(x, y) \ge 0, \quad (x, y) \in \Omega_h
\end{equation*}

and from the discrete max principle:

\begin{equation*}
\max_{(x, y) \in \Omega_h} u_h(x, y) \le \max_{(x, y) \in \Gamma_h} u_h(x, y) = 0
\end{equation*}

Moreover, we have:
\begin{equation*}
\Delta_h u_h(x, y) \le 0, \quad (x, y) \in \Omega_h
\end{equation*}

and then from the discrete min principle:

\begin{equation*}
\min_{(x, y) \in \Omega_h} u_h(x, y) \ge \min_{(x, y) \in \Gamma_h} u_h(x, y) = 0
\end{equation*}

For any $(x, y) \in \Omega_h$ we have: $0 \le u_h(x, y) \le 0$, and so: $u_h(x, y) = 0$. $\blacksquare$

\section{$\infty$-norm and $L^\infty$ norm}

Given a function $f: \mathcal{A} \rightarrow \mathbb{R}$, we set:

\begin{equation*}
||f||_{L^\infty(\mathcal{A})} := \sup_{x \in \mathcal{A}} |f(x)|
\end{equation*}

When A is finite or $\mathcal{A} sub \mathbb{R}^d$ closed and bounded (i.e. compact) then:

\begin{equation*}
||f||_{L^\infty(\mathcal{A})} := \max_{x \in \mathcal{A}} |f(x)|
\end{equation*}

$L^\infty(\mathcal{A})$ is the set of functions $f: \mathcal{A} \rightarrow \mathbb{R}$ such that $||f||_{L^\infty(\mathcal{A})} < +\infty$.

We have, for $v: \overline{\Omega} \rightarrow \mathbb{R}$ of class $C^4$:

\begin{equation*}
\max_{(x, y) \in \Omega_h} {|\Delta_h v|}_{\overline{\Omega_h}} (x, y) - \Delta v(x, y)| \le \frac{h^2}{6} \max\{\max_{(x, y) \in \overline{\Omega}} |\frac{\partial^4 v}{\partial x^4} (x, y)|, \max_{(x, y) \in \overline{\Omega}} |\frac{\partial^4 v}{\partial y^4} (x, y)|\}
\end{equation*}

\begin{equation*}
||\Delta_h v{|}_{\overline{\Omega_h}} (x, y) - \Delta v(x, y)||_{L^\infty(\Omega_h)} \le \frac{h^2}{6} \max\{ ||\frac{\partial^4 v}{\partial x^4}||_{L^\infty(\overline{\Omega_h})}, ||\frac{\partial^4 v}{\partial y^4}||_{L^\infty(\overline{\Omega_h})} \}
\end{equation*}

The stability result says that the map: $(f, g) \rightarrowtail u_h$ (underbraces: data for the discrete problem ... solution of the discrete problem) is uniformly bounded with respect to h, i.e.:

\begin{equation*}
||u_h||_{L^\infty(\overline{\Omega_h})} \le C \cdot \max \{ ||f||_{L^\infty(\Omega)}, ||g||_{L^\infty(\Gamma)} \}
\end{equation*}

with a constant C independent of $f$, $g$ and $h$.

\subsection{Theorem (stability problem)}

Let $f \in L^\infty(\Omega)$ and let $g \in L^\infty(\Gamma)$ and let $u_h$ be the solution of the discrete problem:

\begin{equation*}
\Delta_h u_h = f(x, y), \qquad (x, y) \in \Omega_f
\end{equation*}

\begin{equation*}
u(x, y) = g(x, y) \qquad (x, y) \in \Gamma_h
\end{equation*}

Then we have:

\begin{equation*}
{||u_h||}_{L^\infty(\overline{\Omega})} \le \frac{1}{8} \overbrace{{||f||}_{L^\infty(\Omega)}}^{\le max} + \overbrace{{||g|}_{L^\infty(\Gamma)}}^{\le max}
\end{equation*}


\subsubsection{Proof:}
We introduce the mesh function:

\begin{equation*}
\Phi(x, y) = \frac{(x - \frac{1}{2})^2 + (y - \frac{1}{2})^2}{4} = \frac{d^2}{4}, \quad (x, y) \in \overline{\Omega_h}
\end{equation*}

We have:

\begin{equation*}
\Delta_h \Phi(x, y) = 1, \quad (x, y) \in \Omega_h
\end{equation*}

\begin{equation*}
\Delta_h \Phi(x, y) = \frac{\Phi(x, y-h) + \Phi(x - h, y) - 4 \Phi(x, y) + \Phi(x + h, y) + \Phi(x, y + h)}{h^2}
\end{equation*}

\begin{equation*}
\Phi(x, y - h) = \frac{(x - \frac{1}{2})^2 + (y - h - \frac{1}{2})^2}{4} = \frac{(x - \frac{1}{2})^2 + (y - \frac{1}{2})^2 - 2(y - \frac{1}{2})h + h^2}{4}
\end{equation*}

\begin{equation*}
\Phi(x - h, y) = \frac{(x - \frac{1}{2})^2 + (y - \frac{1}{2})^2 - 2(x - \frac{1}{2})h + h^2}{4}
\end{equation*}

\begin{equation*}
-4 \Phi(x, y) = \frac{-4(x - \frac{1}{2})^2 - 4(y - \frac{1}{2})^2}{4}
\end{equation*}

\begin{equation*}
\Phi(x, y + h) = \frac{(x - \frac{1}{2})^2 + (y - \frac{1}{2})^2 + 2(x - \frac{1}{2})h + h^2}{4}
\end{equation*}

\begin{equation*}
\Phi(x + h, y) = \frac{(x - \frac{1}{2})^2 + (y - \frac{1}{2})^2 + 2(y - \frac{1}{2})h + h^2}{4}
\end{equation*}

which means
\begin{equation*}
\Delta_h \Phi(x, y) = \frac{\frac{4 h^2}{4}}{h^2} = 1
\end{equation*}

We have:

\begin{equation*}
0 \le \Phi(x,y) \le \frac{1}{8}, \quad (x, y) \in \overline{\Omega_h}
\end{equation*}

\begin{equation*}
\Phi(x, y) = \frac{d^2}{4} \le \frac{l^2}{4} = \frac{(\frac{\sqrt{2}}{2})^2}{4} = \frac{(\frac{1}{\sqrt{2}})^2}{4}
\end{equation*}

where $l$ is the diagonal from the center of the square to a corner:

We prove that $\Delta_h$ is a linear operator, i.e. for $v_h, w_h: \overline{\Omega_h} \rightarrow \mathbb{R}$ we have:

\begin{equation*}
\Delta_h(v_h + w_h) = \Delta_h v_h + \Delta_h w_h
\end{equation*}

for $v_h: \overline{\Omega_h} \rightarrow \mathbb{R}$ and $\alpha \in \mathbb{R}$ we have:

\begin{equation*}
\Delta_h \dots
\end{equation*}

For $(x, y) \in \Omega_h$:

\begin{equation*}
\Delta_h(v_h + w_h)(x, y) = ( v_h(x, y-h) + w_h(x, y-h) + v_h(x-h, y) + w_h(x-h, y) - 4 v_h(x,y) - 4 w_h(x, y) + v_h(x, y+h) + w_h(x, y+h) + v_h(x+h, y) + w_h(x+h, y) ) \cdot \frac{1}{h^2} = 
\end{equation*}

\begin{equation*}
= ( v_h(x, y-h)+ v_h(x-h, y) - 4 v_h(x,y) + v_h(x, y+h) + v_h(x+h, y)) \cdot \frac{1}{h^2} + ( w_h(x, y-h) + w_h(x-h, y) - 4 w_h(x, y) + w_h(x, y+h) + w_h(x+h, y) ) \cdot \frac{1}{h^2} = 
\end{equation*}

\begin{equation*}
= \Delta_h v_h(x, y) + \Delta_h w_h(x, y)
\end{equation*}


\begin{equation*}
\Delta_h(\alpha v_h)(x,y) = ( \alpha v_h(x, y-h) + \alpha v_h(x-h, y) - \alpha 4 v_h(x,y) + \alpha v_h(x, y+h) + \alpha v_h(x+h, y)) \cdot \frac{1}{h^2} = \alpha (v_h(x, y)) \dots
\end{equation*}

Set:
\begin{equation*}
A := {||f||}_{L^\infty(\Omega)}, B := {||g||}_{L^\infty(\Gamma)}
\end{equation*}

For $(x, y) \in \Omega_h$:
\begin{equation*}
\Delta_h (u_h + A\Phi) (x, y) = \Delta_h u_h(x, y) + A \underbrace{\Delta_h \Phi(x, y)}_{=1} = \Delta_h u_h(x, y) + A = f(x, y) + A \ge - |f(x, y)| + \underbrace{A}_{= \sup_{(x, y) \in \Omega} |f(x, y)} \ge 0
\end{equation*}

\begin{equation*}
\Delta_h (u_h - A\Phi) (x, y) = \Delta_h u_h(x, y) - A \underbrace{\Delta_h \Phi(x, y)}_{=1} = \Delta_h u_h(x, y) - A = f(x, y) - A \le |f(x, y)| - \underbrace{A}_{= \sup_{(x, y) \in \Omega} |f(x, y)} \le 0
\end{equation*}

The discrete maximum principle says that:
\begin{equation*}
\max_{(x, y) \in \Omega_h} (u_h + A \Phi) (x, y) \le \max_{(x, y) \in \Gamma_h} (u_h + A \Phi) (x, y) 
\end{equation*}

\begin{equation*}
\min_{(x, y) \in \Omega_h} (u_h - A \Phi) (x, y) \ge \max_{(x, y) \in \Gamma_h} (u_h - A \Phi) (x, y) 
\end{equation*}

\begin{equation*}
\max_{(x, y) \in \Omega_h} u_h(x, y) \le \max_{(x, y) \in \Omega_h} (\underbrace{u_h + \overbrace{A \Phi(x, y)}^{\ge 0}}_{= (u_h + A \Phi) (x, y)}) \le
\end{equation*}

\begin{equation*}
\le \max_{(x, y) \in \Gamma_h} (\underbrace{u_h + A \Phi(x, y)}_{= (u_h + A \Phi) (x, y)}) = \max_{(x, y) \in \Gamma_h} (g(x, y) + A \underbrace{\Phi(x,y)}_{\le \frac{1}{8}}) \le \max_{(x, y) \in \Gamma_h}(g(x, y) - \frac{1}{8}A) = \max_{(x, y) \in \Gamma_h} \underbrace{g(x, y)}_{\ge -|g(x, y)|} + \frac{1}{8}A \le - \max_{(x, y) \in \Gamma_h} |g(x, y)| - \frac{1}{8}A \le - \dots
\end{equation*}

The opposite with the $\min$.

We have proved, for $(x, y) \in \overline{\Omega_h}$:

\begin{equation*}
-\frac{1}{8}A - B \le u_h(x, y) \le \frac{1}{8}A + B
\end{equation*}

This is exactly the same of:

\begin{equation*}
|u_h(x, y)| \le \frac{1}{8}A + B
\end{equation*}

In other words:
\begin{equation*}
||u_h||_{L^\infty()} \dots
\end{equation*}

\begin{equation*}
|| \Delta_h v|_{\Omega_h} - (\Delta_h v)|_{\Omega_h} ||_{L^\infty(\Omega_h)}
\end{equation*}

\begin{equation*}
|| u_h - u|_{\overline{\Omega_h}}  ||
\end{equation*}

\subsection{Theorem (the convergence theorem)}

Let $u$ be the solution of:

\begin{equation*}
\Delta u(x, y) = f(x, y) \quad (x, y) \in \Omega
\end{equation*}

\begin{equation*}
u(x, y) = g(x, y) \quad (x, y) \in \Gamma
\end{equation*}

and let $u_h$ be the solution of:
\begin{equation*}
\Delta u_h(x, y) = f(x, y) \quad (x, y) \in \Omega_h
\end{equation*}

\begin{equation*}
u_h(x, y) = g(x, y) \quad (x, y) \in \Gamma_h
\end{equation*}

We have:

\begin{equation*}
|| u_h - u|_{\Omega_h} ||_{L^\infty(\overline{\Omega_h})} \le \frac{1}{8} ||\Delta_h u|_{\Omega_h} - (\Delta u)|_{\Omega_h} ||_{L^\infty(\Omega_h)}
\end{equation*}

\subsubsection{Proof:}

We set:
\begin{equation*}
l_h := u_h - u|_{\overline{\Omega_h}}
\end{equation*}

We have:
\begin{equation*}
\Delta_h l_h = \Delta_h (u_h - u|_{\overline{\Omega_h}}) = \Delta_h u_h - \Delta_h u|_{\overline{\Omega_h}} = \Delta_h u_h - (\Delta u)|_{\Omega_h} + (\Delta u)|_{\Omega_h} - \Delta_h u|_{\Omega_h}
\end{equation*}

We have, for $(x, y) \in \Omega_h$
\begin{equation*}
\Delta_h u_h(x, y) - \Delta u (x, y) = g(x, y) - f(x, y) = 0
\end{equation*}

We have:
\begin{equation*}
\Delta_h l_h = - \epsilon_h
\end{equation*}

where
\begin{equation*}
\epsilon_h = \Delta_h u|_{\Omega_h} - (\Delta u)|_{\Omega_h}
\end{equation*}

For $(x, y) \in \Gamma_h$:
\begin{equation*}
l_h (x, y) = u_h (x, y) - u(x, y) = \dots
\end{equation*}

By concluding, we can say that the important error $l_h$ satisfies the discrete problem:
\begin{equation*}
\Delta_h \epsilon_h (x, y) = - \epsilon_h(x, y) \qquad (x, y) \in \Omega_h
\end{equation*}

\begin{equation*}
l_h(x, y) = 0 \qquad (x, y) \in \Gamma_h
\end{equation*}

By means of the previous theorem:
\begin{equation*}
||l_h||_{L^\infty(\overline{\Omega_h})} \le \frac{1}{8} \underbrace{|| -\epsilon_h||_{L^\infty(\Omega_h)}}_{= || \epsilon_h ||_{L^\infty(\Omega_h)} } + \underbrace{||o||_{L^\infty(\Gamma_h)}}_{= 0} \le \frac{1}{8} ||\epsilon_h||_{L^\infty(\Omega_h)} + \dots
\end{equation*}
$\blacksquare$

We know that:
\begin{equation*}
|| \Delta_h u|_{\Omega_h} - (\Delta u)|_{\Omega_h} ||_{L^\infty(\Omega_h)} \le \frac{h^2}{6} \max \{ ||\frac{\partial^4 u}{\partial x^4}||_{L^\infty(\overline{\Omega_h})}, ||\frac{\partial^4 v}{\partial y^4}||_{L^\infty(\overline{\Omega_h})} \}
\end{equation*}

and then:
\begin{equation*}
|| u_h - u|_{\overline{\Omega_h}} ||_{L^\infty(\overline{\Omega_h})} \le \frac{h^2}{48} \max \{ ||\frac{\partial^4 u}{\partial x^4}||_{L^\infty(\overline{\Omega_h})}, ||\frac{\partial^4 v}{\partial y^4}||_{L^\infty(\overline{\Omega_h})} \}
\end{equation*}
when $u \in C^4$.

\subsection{Some final remarks}

Consider:
\begin{itemize}
	\item $u$ is the solution of the continuous problem;
	\item $u_h$ is the solution of the discrete problem;
	\item $\Delta_h$ is a discrete Laplacian, not necessarily the five-point discretization.
\end{itemize}

We define:
\begin{equation*}
l_h := u_h - u|_{\overline{\Omega_h}}
\end{equation*}

$l_h$ is called the \textit{convergence error}:
\begin{equation*}
\epsilon_h := \Delta_h u|_{\Omega_h} - (\Delta u)|_{\Omega_h}
\end{equation*}

$\epsilon_h$ is called the \textit{consistency error}.

We are interested on $l_h$, since $\epsilon_h$ is a "surrogate" of $l_h$.\\


We say that, given $p$ a positive integer:

\begin{itemize}
	\item the discretization is \textit{consistent of order $p$} if:
		\begin{equation*}
		|| \epsilon_h ||_{L^\infty(\Omega_h)} = O(h^p) \qquad N \rightarrow \infty
		\end{equation*}
	\item the discretization is \textit{convergent of order $p$} if
		\begin{equation*}
		|| l_h ||_{L^\infty(\overline{\Omega_h})} = O(h^p) \qquad N \rightarrow \infty
		\end{equation*}
		
	\item the discretization is \textit{stable} if the solution $v_h$ of the discrete problem:
		\begin{equation*}
		\Delta_h v_h(x, y) = l(x,y), \qquad (x, y) \in \Omega_h
		\end{equation*}
		\begin{equation*}
		v_h(x, y) = 0
		\end{equation*}
	
		we have:
		\begin{equation*}
		|| v_h ||_{L^\infty(\overline{\Omega_h})} \le C || l ||_{L^\infty(\Omega)}
		\end{equation*}
		
		where $C$ is independent of $l$ and $h$.
\end{itemize}

We have: consistency of order $p$ and stability $\implies$ convergence of order $p$.

We have:
\begin{equation*}
\Delta_h \l_h = \epsilon_h in \Omega_h
\end{equation*}

\begin{equation*}
l_h = 0 in \Gamma_h
\end{equation*}

If $||\epsilon_h||_{L^\infty(\Omega_h)} = O(h^p)$ (consistency) and $||l_h||_{L^\infty(\overline{\Omega_h})} \le C || \epsilon_h ||_{L^\infty(\Omega_h)}$ then $||L_h||_{L^\infty(\overline{\Omega_h})} = O(h^p)$ (convergence).

