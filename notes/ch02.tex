\chapter{Numerical methods for PDE}

PDE stands for Partial Differential Equations, and we are going to study in particular Finite Difference Methods, not Finite Element Methods.

PDEs can be divided in three categories: elliptic, parabolic and high-parabolic. We will see a test problem on each of these categories.

\section{Poisson 2D problem with Dirichlet boundary conditions}
\begin{equation}
\text{Poisson equation:\quad} \Delta u(x, y) = f(x, y), (x, y) \in \Omega
\end{equation}

\begin{equation}
\text{Dirichlet Boundary condition:\quad} u(x, y) = g(x, y), (x, y) \in \Gamma
\end{equation}

where $\Omega = (0, 1)^2$

% im:I

Example: the electric potential inside a domain with a defined potential inside, $f$, and a fixed potential, $g$, at the border.

The Poisson 2D problem can be solved analytically using \textit{separation of variables} by searching for solutions of this type:

\begin{equation}
	u(x, y) = X(x) Y(y)
\end{equation}

Discretization: fix N positive integer and put $h = \frac{1}{N}$ as the stepsize.

Now we discretize the plane $\mathbb{R}^2$ by the mesh:

\begin{equation}
	\mathbb{R}^2_h = \{(mh, nh): m,n \in \mathbb{Z}\}
\end{equation}

% im:2

For a mesh point $(mh, nh)$, with $m, n \in \mathbb{Z}$ we can define four neighbors:

% im:3

The discrete version of $\Omega$ is


\begin{equation}
	\Omega_h := \Omega \cap \mathbb{R}^2_h
\end{equation}

% 4

while the discrete version of $\Gamma$ is the set of points $\Gamma_h$ in $\mathbb{R}^2_h$ which are not in $\Omega_h$, but have a neighbor in $\Omega_h$.

Notice how $\Gamma_h \neq \Gamma \cap \mathbb{R}^2_h$ because the four corners $\{(0,0), (0,1), (1,0), (1,1)\}$ are not included. 

We also define $\overline{\Omega_h} = \Omega_h \cup \Gamma_h$: this is the discrete version of $\overline{\Omega_h}  = \Omega \cup \Gamma$.\\


The discrete version of the Poisson problem with Dirichlet Boundary conditions is:


\begin{equation}
\text{discrete Poisson equation:\enspace} \Delta_h u_h(x, y) = f(x, y), (x, y) \in \Omega_h
\end{equation}
	
\begin{equation}
\text{discrete Dirichlet Boundary condition:\enspace} u_h(x,y) = g(x, y), (x, y) \in \Gamma_h
\end{equation}


An operator is a map/function, whose domains and codomains are set of functions.

$\Delta_h$ is a discrete Laplacian, i.e. and operator such that:
\begin{itemize}
	\item it associates (as a map) to a mesh function $\overline{\Omega_h} \rightarrow \mathbb{R}$ an interior mesh function $\Omega_h \rightarrow \mathbb{R}$;
	
	\item it approximates the Laplacian, in the sense that, for function $v: \overline{\Omega_h} \rightarrow \mathbb{R}$ sufficiently smooth, which means $C^2(\overline{\Omega_h})$:
		
		\begin{equation}
			\Delta_h v |_{\overline{\Omega_h}} \; (x,y) \approx \Delta v(x, y) \quad (x, y) \in \Omega_h
		\end{equation}

	where $|_{\overline{\Omega_h}}$ is the restriction of $v$ to $\overline{\Omega_h}  \subseteq \overline{\Omega}$.	
\end{itemize}

We are going to show how to construct a discrete Laplacian.\\


Consider a function $v$ of only one variable $t$. We discretize $v''(t)$ by:

\begin{equation}
\frac{v(t - h) - 2 v(t) + v(t + h)}{h^2}
\end{equation}
.

%5

\begin{equation}
\frac{\frac{v(t + h) - v(t)}{h} - \frac{v(t) - v(t - h)}{h}}{h} \approx\\
\frac{v'(t) - v'(t-h)}{h} \approx\\
v''(t-h) \approx v''(t)
\end{equation}

By Taylor approximations under the assumptions that v is $C^4$:

\begin{equation}
v(t-h) = v(t) - v'(t) h + \frac{1}{2} v''(t) h^2 - \frac{1}{6} v'''(t) h^3 + \frac{1}{24} v^{(4)}(\alpha_h) h^4
\label{t-h}
\end{equation}

where $\alpha_h \in [t-h, t]$;

\begin{equation}
v(t+h) = v(t) + v'(t) h + \frac{1}{2} v''(t) h^2 + \frac{1}{6} v'''(t) h^3 + \frac{1}{24} v^{(4)}(\beta_h) h^4
\label{t+h}
\end{equation}

where $\beta_h \in [t, t+h]$.

By summing \ref{t-h} and \ref{t+h} we obtain:
\begin{equation}
v(t-h) + v(t+h) = 2 v(t) + v''(t) h^2 + \frac{1}{24} (v^{(4)}(\alpha_h) + v^{(4)}(\beta_h)) h^4
\end{equation}

\begin{equation}
v(t-h) - 2 v(t) + v(t+h) = v''(t) h^2 + \frac{1}{24} (v^{(4)}(\alpha_h) + v^{(4)}(\beta_h)) h^4
\end{equation}
\begin{equation}
\frac{v(t-h) - 2 v(t) + v(t+h)}{h^2} = v''(t) + \frac{1}{24} (v^{(4)}(\alpha_h) + v^{(4)}(\beta_h)) h^2
\end{equation}

So we got an approximation of the second derivative of the function $v$.

\subsection{Discretization of the Laplacian}

For a smooth function $v: \overline{\Omega_h} \rightarrow \mathbb{R}$ we approximate the Laplacian:

% 6

by:

\begin{equation}
\Delta_h v(x, y) = \frac{v(x-h, y) - 2v(x,y) + v(x+h, y)}{h^2} + \frac{v(x, y-h) - 2v(x,y) + v(x, y+h)}{h^2}
\end{equation}

at $(x, y) \in \Omega_h$.

% 7

The computation of $\Delta_h v(x, y)$ requires $v(x-h, y)$ and $v(x, y+h)$ with $(x-h, y), (x, y+h) \in \Gamma_h$.

This discretization of the Laplacian is called the \textit{Five-points discretization}:

% 8


\subsection{Assess approximation of the discrete Laplacian}

Consider $v: \overline{\Omega} \rightarrow \mathbb{R}$ of class $C^4$. For $(x, y) \in \Omega_h$:

\begin{equation}
|\Delta_h v {|}_{\overline{\Omega_h}} (x,y) \Delta v(x,y)|
\end{equation}

\begin{equation}
|\frac{v(x-h, y)  - 2v(x, y) + v(x+h, y)}{h} + \frac{v(x, y-h) - 2v(x, y) + v(x, y+h)}{h} - \frac{\partial^2 v}{\partial x^2}(x, y) - \frac{\partial^2 v}{\partial y^2}(x, y)|
\end{equation}

which is $\leq$ than: 

\begin{equation}
|\frac{v(x-h, y) - 2v(x, y) + v(x+h, y)}{h^2} - \frac{\partial^2 v}{\partial x^2}(x, y)| + |\frac{v(x, y-h) - 2v(x, y) + v(x, y+h)}{h^2} - \frac{\partial^2 v}{\partial y^2}(x, y)|
\label{buh}
\end{equation}
.

\ref{buh} is equal to:

\begin{equation}
|\frac{1}{24} ( \frac{\partial^4 v}{\partial x^4}(\alpha_h, y) + \frac{\partial^4 v}{\partial x^4}(\beta_h, y) \; h^2) | + |\frac{1}{24} ( \frac{\partial^4 v}{\partial y^4}(x, \gamma_h) + \frac{\partial^4 v}{\partial y^4}(x, \delta_h) \; h^2) |
\label{buh2}
\end{equation}

where $\alpha_h \in [x-h, x], \beta_h \in [x, x+h], \gamma_h \in [y-h, y], \delta_h \in [y, y+h]$.

\ref{buh2} (the error) can be bounded ($\leq$) by:

\begin{equation}
\frac{1}{6} \max\{ \max_{(x, y) \in \overline{\Omega}} |\frac{\partial^4 v}{\partial x^4} (x, y)|, \max_{(x, y) \in \overline{\Omega}} |\frac{\partial^4 v}{\partial y^4} (x, y)| \} \; h^2
\end{equation}

So we can conclude that:

\begin{equation}
\max_{(x,y) \in \Omega_h}  |\Delta_h v {|}_{\overline{\Omega_h}} (x,y) \Delta v(x,y)| \leq \frac{1}{6} \max\{ \max_{(x, y) \in \overline{\Omega}} |\frac{\partial^4 v}{\partial x^4} (x, y)|, \max_{(x, y) \in \overline{\Omega}} |\frac{\partial^4 v}{\partial y^4} (x, y)| \} \; h^2
\end{equation}


This is called the \textit{consistency error} of the five point approximation of the discrete Laplacian.

\subsection{Exercise}

Given $v \in C^6(\Omega)$, find  a function $C(x,y)$ of the point $(x, y) \in \overline{\Omega}$ and $M \ge 0$ such that 
\begin{equation*}
\max_{(x,y) \in \Omega_h}  |\Delta_h v {|}_{\overline{\Omega_h}} (x,y) - \Delta v(x,y) - C(x, y) h^2| \le M h^4
\end{equation*}

\begin{equation*}
%\max_{(x,y) \in \Omega_h}  |
\Delta_h v {|}_{\overline{\Omega_h}} (x,y) = \Delta v'(x,y) + \text{error}
\end{equation*}

with $\text{error} = O(h^2)(*) = C(x, y) h^2 + O(h^4) (**)$.


Remind that:
\begin{equation}
v''(t) \approx \frac{v(t - h) - 2 v(t) + v(t + h)}{h^2}
\end{equation}

we can continue with the Taylor expansion:
\begin{equation*}
v(t-h) = v(t) - v'(t) h + \frac{v''(t)}{2} h^2 - \frac{v'''(t)}{6} h^3 + \frac{v^{4}(t)}{24} h^4  - \frac{v^{5}(t)}{120} h^5 + \frac{v^{6}(t)}{6!} h^6
\end{equation*}

\begin{equation*}
v(t-h) = v(t) + v'(t) h + \frac{v''(t)}{2} h^2 + \frac{v'''(t)}{6} h^3 + \frac{v^{4}(t)}{24} h^4  + \frac{v^{5}(t)}{120} h^5 + \frac{v^{6}(t)}{6!} h^6
\end{equation*}

where $\alpha \in [t-h, t]$ and $\beta \in [t, t+h]$.

\begin{equation*}
v(t-h) + v(t+h) = 2v(t) + v''(t) h^2 + \frac{v^{4}(t)}{12} h^4 + \frac{v^{6}(\alpha)}{6!} h^6 + \frac{v^{6}(\beta)}{6!} h^6
\end{equation*}

\begin{equation*}
\frac{v(t-h) - 2v(t)+ v(t+h)}{h^2} - v''(t) h^2 = \frac{v^{4}(t)}{12} h^4 + \frac{v^{6}(\alpha)}{6!} h^6 + \frac{v^{6}(\beta)}{6!} h^6
\end{equation*}


\dots


The discrete problem:
\begin{equation*}
\text{Poisson equation:\quad} \Delta u_h(x, y) = f(x, y), (x, y) \in \Omega_h
\end{equation*}

\begin{equation*}
\text{Dirichlet Boundary condition:\quad} u_h(x, y) = g(x, y), (x, y) \in \Gamma_h
\end{equation*}

is a linear system of $(N-1)^2$ unknowns (where $N$ is the number of subdivisions we do in each dimension of the square)  and $(N-1)^2$ equations.

The unknowns are $u_{ij} := u_h (ih, jh)$ with $i \in \{1, \dots, N-1 \}$ and $j \in \{1, \dots, N-1 \}$.

% 1 : 2018/03/15

.. we don't consider the values on the borders as unknowns because they are known, from the second equation of the Dirichlet Boundary condition.

On the border $\Gamma_h$:

\begin{equation*}
u_{ij} = u_h (ih, jh) = h_{ij} := g(ih, jh)
\end{equation*}

where 
\begin{equation*}
(i,j) \in \{0\} \times \{1, \dots, N-1 \} \cup \{1, \dots, N-1 \} \times \{0\} \cup \{N\} \times  \{1, \dots, N-1 \} \cup  \{1, \dots, N-1 \} \times \{N\}
\end{equation*}

The equations are 
\begin{equation*}
\Delta_h u_h (x, y) = f(x, y), \quad \in \Omega_h
\end{equation*}

each equation corresponds to a point in $\Omega_h$ with $(N-1)^2$ equations.

The equation corresponding to the point $(ih, jh)$, with $i \in  \{1, \dots, N-1 \}, j \in  \{1, \dots, N-1 \}$ is:

\begin{equation*}
\frac{u_{i-1\; j} - 2 u_{ij} + u_{i+1\; j}}{h^2} + \frac{u_{i\; j-1} - 2 u_{ij} + u_{i\; j+1}}{h^2} = f(ih, jh) = f_{ij}
\end{equation*}

\begin{equation*}
u_{i\; j-1} + u_{i-1\; j} -4 u_{ij} +  u_{i+1\; j} + u_{i\; j+1} = h^2 f_{ij}
\end{equation*}

In order to apply a linear system of the form $A u = b$, we need to re-arrange the equations of the points of the square (which are 2D) into a 1D vector.

Consider $N=4$:

% 2
\dots

We want to rewrite our matrix of unknowns $u$:

\begin{equation*}
u = \begin{bmatrix}
u_{11} & u_{21} & u_{31} \\
u_{12} & u_{22} & u_{32} \\
u_{13} & u_{23} & u_{33} \\
\end{bmatrix}
\end{equation*}

as a vector:

\begin{equation*}
u = \begin{bmatrix}
u_{11} \\
u_{21} \\
u_{31} \\
u_{12} \\
u_{22} \\
u_{32} \\
u_{13} \\
u_{23} \\
u_{33} \\
\end{bmatrix}
\end{equation*}

Then we can rewrite our equation $u_{i\; j-1} + u_{i-1\; j} -4 u_{ij} +  u_{i+1\; j} + u_{i\; j+1} = h^2 f_{ij}$ as (for the point $(1,1)$):

\begin{equation*}
u_{10} + u_{01} - 4u_{11} + u_{21}  + u_{12} = h^2 f_{11}
\end{equation*}

where $u_{10}$ and $u_{01}$ correspond to $g_{10}$ and $g_{01}$ respectively.

\begin{equation*}
- 4u_{11} + u_{21}  + u_{12} = h^2 f_{11} - u_{10} - u_{01}
\end{equation*}

For the point $(2,1)$:

\begin{equation*}
u_{20} + u_{11} - 4u_{21} + u_{31}  + u_{22} = h^2 f_{22}
\end{equation*}

rewritten as:
\begin{equation*}
u_{11} - 4u_{21} + u_{31} + u_{22} = h^2 f_{22} - g_{20}
\end{equation*}

We can write a matrix $\mathcal{A}$:

\begin{equation*}
\mathcal{A} = \begin{bmatrix}
-4 &  1 &  0 &  1 &  0 &  0 &  0 &  0 &  0 \\
 1 & -4 &  1 &  0 &  1 &  0 &  0 &  0 &  0 \\
 0 &  1 & -4 &  0 &  0 &  1 &  0 &  0 &  0 \\
 1 &  0 &  0 & -4 &  1 &  0 &  1 &  0 &  0 \\
 0 &  1 &  0 &  1 & -4 &  1 &  0 &  1 &  0 \\
 0 &  0 &  1 &  0 &  1 & -4 &  0 &  0 &  1 \\
 0 &  0 &  0 &  1 &  0 &  0 & -4 &  1 &  0 \\
 0 &  0 &  0 &  0 &  1 &  0 &  1 & -4 &  1 \\
 0 &  0 &  0 &  0 &  0 &  1 &  0 &  1 & -4 \\
\end{bmatrix}
\end{equation*}

and the unknown vector $u$:

\begin{equation*}
u = \begin{bmatrix}
h^2 f_{11} - g_{01} - g_{10} \\
h^2 f_{21} - g_{20}          \\
h^2 f_{31} - g_{14}          \\
h^2 f_{12} - g_{02}          \\
h^2 f_{22}                   \\
h^2 f_{32} - g_{24}          \\
h^2 f_{13} - g_{03} - g_{14} \\
h^2 f_{23} - g_{24}          \\
h^2 f_{33} - g_{34} - g_{43} \\
\end{bmatrix}
\end{equation*}

$\mathcal{A}$ can be rewritten as a $3x3$ blocks matrix with blocks of size $3x3$:
\begin{equation*}
\mathcal{A} = \begin{bmatrix}
  A & I_3 &   0 \\
I_3 &   A & I_3 \\
  0 & I_3 &   A \\
\end{bmatrix}
\end{equation*}


For a general $N$, $\mathcal{A}$ is a $(N - 1) \times (N - 1)$ block matrix:
\begin{equation*}
\mathcal{A} = \begin{bmatrix}
      A & I_{N-1} &        &         &         \\
I_{N-1} &       A & \ddots &         &         \\
        &  \ddots & \ddots &  \ddots &         \\
%
        &         & \ddots &       A & I_{N-1} \\
        &         &        & I_{N-1} &       A \\
\end{bmatrix}
\end{equation*}

% -4  1
%  1 -4


A generic matrix with dimension $n \times n$ is called ``sparse" if the number of non-zero elements is $O(n)$.

The matrix $\mathcal{A}$ is sparse, block tridiagonal (it has elements on the diagonal, on the diagonal of the upper matrix and on the diagonal of the lower matrix), symmetric and \dots.

\begin{equation*}
\end{equation*}


\begin{equation*}
\end{equation*}


\begin{equation*}
\end{equation*}


\begin{equation*}
\end{equation*}


\begin{equation*}
\end{equation*}


\begin{equation*}
\end{equation*}


\begin{equation*}
\end{equation*}


\begin{equation*}
\end{equation*}


\begin{equation*}
\end{equation*}


\begin{equation*}
\end{equation*}


\begin{equation*}
\end{equation*}


\begin{equation*}
\end{equation*}


\begin{equation*}
\end{equation*}


\begin{equation*}
\end{equation*}


\begin{equation*}
\end{equation*}


\begin{equation*}
\end{equation*}


\begin{equation*}
\end{equation*}


\begin{equation*}
\end{equation*}


\begin{equation*}
\end{equation*}


\begin{equation*}
\end{equation*}


\begin{equation*}
\end{equation*}


\begin{equation*}
\end{equation*}


\begin{equation*}
\end{equation*}


\begin{equation*}
\end{equation*}


\begin{equation*}
\end{equation*}


\begin{equation*}
\end{equation*}


\begin{equation*}
\end{equation*}


\begin{equation*}
\end{equation*}