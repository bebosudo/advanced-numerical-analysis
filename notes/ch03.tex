\section{Fourier Analysis}

$L(\Omega_h)$ is the set of functions $\Omega_h \rightarrow \mathbb{R}$; $L(\Omega_h)$ is isomorphic, as $\mathbb{R}^{(N-1)^2}$.

The function in $L(\Omega_h)$ are extended to $\overline{\Omega_h}$ by setting value zero there.

So, we can now consider the Laplacian $\Delta_h$ as an operator $L(\Omega_h) \rightarrow L(\Omega_h)$.

Is this operator invertible? Or, for $f_h \in L(\Omega_h)$, there exists a unique function $v_h$ such that:
\begin{equation*}
\Delta_h v_h = f_h
\end{equation*}

\begin{equation}
\Delta_h v_h(x, y) = f_h(x, y)  \quad (x, y) in \Omega_h
\end{equation}

\begin{equation}
v_h(x, y) = 0 \quad (x, y) \in \Gamma_h
\end{equation}

This means $\Delta_h$ invertible.

We know that the previous problem has a unique solution.

So $\Delta_h$ is invertible and $\Delta_h^{-1}$ exists.
Is $\Delta_h^{-1}$ uniformly bounded with respect to $h$?

Consider $L(\Omega_h)$ with the $L^\infty(\Omega_h)$.

\begin{equation*}
|| \Delta_h^{-1} || = \sup_{f_h \in L^{\infty}(\Omega_h), f_h \neq 0} \frac{||v_h||_{L^{\infty}(\Omega_h)}}{||f_h||_{L^{\infty}(\Omega_h)}}
\end{equation*}

$\Delta_h^{-1}$ uniformly bounded with respect to $h$ means there exists $C \ge 0$, independent of $h$, such that:

\begin{equation*}
|| \Delta_h^{-1} || \le C
\end{equation*}

For the five-point discretization, we have:

\begin{equation*}
||v_h||_{L^{\infty}(\Omega_h)} \le \frac{1}{8} ||f_h||_{L^{\infty}(\Omega_h)}
\end{equation*}

for any problem with right-hand side $f_h$.

Then:

\begin{equation*}
|| \Delta_h^{-1} || \le \underbrace{\frac{1}{8}}_{= C}
\end{equation*}

The relation between the convergence error:

\begin{equation*}
e_h = u_h - {u|}_{\Omega_h} \in L(\Omega_h)
\end{equation*}

and the consistency error:

\begin{equation*}
\epsilon_h = \Delta_h u|_{\overline{\Omega_h}} - (\Delta u)|_{\Omega_h} \in L(\Omega_h)
\end{equation*}

is:
\begin{equation*}
\Delta_h e_h = - \epsilon_h
\end{equation*}

So:
\begin{equation*}
e_h = \Delta_h^{-1}(-\epsilon_h)
\end{equation*}

and then:
\begin{equation*}
||e_h||_{L^{\infty}(\Omega_h)} \le || \Delta_h^{-1} || {|| \epsilon_h ||}_{L^\infty(\Omega_h)}
\end{equation*}

and then:
\begin{equation*}
||e_h||_{L^{\infty}(\Omega_h)} \le C {|| \epsilon_h ||}_{L^\infty(\Omega_h)}
\end{equation*}

$L^\infty$ on $L(\Omega_h)$ is the discrete analogous of the $L^\infty$ on $L(\Omega)$.

We introduce a norm on $L(\Omega_h)$ that is a discrete analogous of the $L^2$ norm on $L^2(\Omega)$.

For $v: \Omega \rightarrow \mathbb{R}$,

\begin{equation*}
|| v || = \sqrt{\int_{(x, y) \in \Omega} v(x,y)^2 d(x, y)}
\end{equation*}

is called the $L^2$ norm of $v$.

$L^2(\Omega)$ is the set of the function $\Omega \rightarrow \mathbb{R}$ such that:

\begin{equation*}
\int_{(x, y) \in \Omega} v(x,y)^2 d(x, y) < \infty
\end{equation*}

On $L^2(\Omega)$, we can introduce the scalar product:
\begin{equation*}
<v, w> = \int_{(x, y) \in \Omega} v(x,y) w(x,y) d(x, y)
\end{equation*}
with $v, w \in L^2(\Omega)$.

Then, the $L^2$ norm is the norm derived by this scalar product:
\begin{equation*}
||v|| = \sqrt{<v, v>}, \quad L^2(\Omega)
\end{equation*}

Now, we introduce discrete forms of this scalar product and this norm on $L(\Omega_h)$ and then to bound uniformly with respect to $h$ the operator norm:

\begin{equation*}
|| \Delta_h^{-1} || = \sup_{f_h \in L^{\infty}(\Omega_h), f_h \neq 0} \frac{||v_h||_h}{||f_h||_h}
\end{equation*}

where $||.||_h$ is the discrete $L^2$ norm.

First, we consider the $1D$ version of the Poisson problem.

\begin{equation*}
\Omega = (0, 1) =: I
\end{equation*}
\begin{equation*}
\Omega_h = \{h, 2h, \dots, (N-1)h \} =: I_h
\end{equation*}

$L^2(I)$ is the set of the functions $v: I \rightarrow \mathbb{R}$ such that:

\begin{equation*}
\int_{0}^{1} v(x)^2 d(x) < +\infty
\end{equation*}

The scalar product on $L^2(I)$ is:
\begin{equation*}
<v, w> = \int_0^1 v(x) w(x) d(x), \quad v, w \in L^2(I)
\end{equation*}

and the $L^2$ norm is:
\begin{equation*}
||v|| = \sqrt{\int_0^1 v(x)^2 d(x)}, \quad v \in L^2(I)
\end{equation*}

On $L(I_n)$ we introduce the scalar product:
\begin{equation*}
<v_h, w_h> = h \sum_{k=1}^{N-1} v_h(k h) w_h(kh)
\end{equation*}

and the norm:
\begin{equation*}
{|| v_h ||}_h = \sqrt{<v_h, v_h>} = \sqrt{h \sum_{k=1}^{N-1} v_h(k h)^2}, \quad v_h \in L(I_h)
\end{equation*}

$||.||$ is the $L^2$ norm on $I_h$.

\subsubsection{Exercise}
Explain why the previous scalar product and norm on $L(I_h)$ are discretizations of the scalar product and norm on $L^2(I)$.

\begin{equation*}
\int_{0}^{1} g(x) dx \approx h \sum_{k=1}^{N-1} g(kh) = \sum_{k=1}^{N-1} g(kh) h
\end{equation*}

When $g = v w \rightarrow $ scalar product, $g = v^2 \rightarrow L^2$ norm.

In $L^2(I)$ we have Fourier series. Consider the functions $\Phi_m \in L^2(I)$, $m \in \{1, 2, 3, \dots \}$ given by:

\begin{equation*}
\Phi_m(x) = sin(m \pi x), \quad x \in (0, 1)
\end{equation*}

These functions are orthogonal:
\begin{equation*}
<\Phi_m, \Phi_n> = 0 \quad \forall m \neq n
\end{equation*}

They constitute an orthogonal base for $L^2(I)$; for any $v \in L^2(I)$ we have the \textit{Fourier series of $v$}:
\begin{equation*}
v = \sum_{m=1}^{\infty} c_m \Phi_m
\end{equation*}

with:
\begin{equation*}
c_m = \frac{<v, \Phi_m>}{<\Phi_m, \Phi_m>}, \quad m \in \{1, 2, 3, \dots \}
\end{equation*}

The Fourier series converges in $L^2$:
\begin{equation*}
\lim_{M \rightarrow \infty} || v - \sum_{m = 1}^{M} c_m \Phi_m || = 0
\end{equation*}

We have the \textit{Parseval's identity}:
\begin{equation*}
|| v ||^2 = \sum_{m = 1}^\infty c_m^2 || \Phi_m ||^2, \quad v \in L^2(I)
\end{equation*}

The functions $\Phi_m$, with $m \in \{1, 2, 3, \dots\}$ are eigenvectors of the 1D Laplacian: $\Delta$ = "second derivative".

For $m \in \{1, 2, 3, \dots \}$, we have:
\begin{equation*}
\Delta \Phi_m(x) = \frac{d^2}{d x^2} \sin(m \pi x) = \frac{d}{dx}(m \pi \cos(m \pi x)) = m \pi (-m \pi \sin(m \pi x)) = -m^2 \pi^2 \sin(m \pi x)
\end{equation*}
\begin{equation*}
= - m^2 \pi^2 \Phi_m(x), \quad x \in (0, 1)
\end{equation*}

$\Phi_m$ is an eigenvector of $\Delta$ with relevant eigenvalue $- m^2 \pi^2 = \lambda_m$.

We do the same for $L(I_h)$.

We introduce the functions $\Phi_{m,h} \in L(I_h), m \in \{1, 2, \dots \}$, given by:
\begin{equation*}
\Phi_{m,h}(x) = \sin(m \pi x), \quad x \in I_h
\end{equation*}

The functions $\Phi_{m,h}, m \in \{1, 2, \dots \}$, are eigenvectors of the discrete laplacian:
\begin{equation*}
\Delta_h v_h(x) = \frac{v_h(x-h) - 2v_h(x) + v_h(x + h)}{h^2}, \quad x \in I_h and v_h \in L(I_h)
\end{equation*}

For $m \in \{1, 2, 3, \dots \}$,
\begin{equation*}
\Delta_h \Phi_{m,h} (x) = \frac{\sin(m \pi (x-h)) - 2 \sin (m \pi x) + \sin(m \pi (x + h))}{h^2}
\end{equation*}

Using Prostapheresis formulas:
\begin{equation*}
= \frac{2 \sin(m \pi x) \cos(m \pi x) - 2 \sin(m \pi x)}{h^2}
\end{equation*}

\begin{equation*}
= - \frac{1 - 2 \cos(m \pi x)}{h^2} \underbrace{\sin(m \pi x)}_{\Phi_{m,h}(x)}, \quad x \in I_h
\end{equation*}

$\Phi_{m,h}$ is an eigenvector of the discrete 1D laplacian with relevant eigenvalue:

$\lambda_{m, h} = -2 \frac{1 - \cos(m \pi h)}{h^2} = -2 \frac{2 \sin^2(\frac{m \pi x}{2})}{h^2} = -4 \frac{\sin^2(\frac{m \pi x}{2})}{h^2}$ 

We have:
\begin{equation*}
0 > \lambda_{1, h} > \lambda_{2, h} > \dots > \lambda_{N - 1, h}
\end{equation*}

\subsubsection{Exercise}

Prove that:
\begin{enumerate}
	\item $\lim_{h \rightarrow 0} \lambda_m = - m^2 \pi^2$
	\item $\lambda_{N - 1, h} > - \frac{4}{h^2}$
	\item $- 8 \ge \lambda_{1,h}$
\end{enumerate}

Proofs:
\begin{enumerate}
	\item 
\begin{equation*}
\lim_{h \rightarrow 0} \lambda_m = \lim_{h \rightarrow 0}( -4 \frac{\sin^2(\frac{m \pi x}{2})}{h^2})
\end{equation*}
	Using the limit $\lim_{x \rightarrow 0} \frac{\sin x}{x} = 1$ and $\lim_{x \rightarrow 0} \frac{\sin^2 x}{x^2} = 1$
\begin{equation*}
= -m^2 h^2 \lim_{h \rightarrow 0} (\frac{\sin(\frac{m \pi x}{2})}{\frac{m \pi x}{2}})^2 = - m^2 h^2
\end{equation*}
	
	\item
\begin{equation*}
\lambda_{N - 1, h} = - 4 \frac{\sin^2(\frac{(N-1) \pi x h}{2})}{h^2} > -4 \frac{\sin^2(\pi/2)}{h^2} = - \frac{4}{h^2}
\end{equation*}

	\item 
\begin{equation*}
\lambda_{1,h} = - 4 \frac{\sin^2(\frac{\pi h}{2})}{h^2} = - \pi^2 ( \frac{\sin (\frac{\pi h}{2})}{\frac{\pi h}{2}})^2
\end{equation*}
As a function of $h$, $\lambda_{1, h}$ obtains the maximum value when $h$ is the maximum.

The maximum $h$ possible is $h = \frac{1}{2}$.

\begin{equation*}
\lambda_{1,h} \le \lambda_{1, \frac{1}{2}} = - 4 \frac{\sin^2(\pi \frac{1}{2})}{(\frac{1}{2})^2} = -4 \frac{\sin^2(\frac{\pi}{4})}{\frac{1}{4}} = - 16 (\frac{\sqrt{2}}{2})^2 = -8
\end{equation*}

\end{enumerate}

$\blacksquare$

$\Phi_{1,h}, \dots, \Phi_{N-1,h}$ are a bases for the space $L(I_h) = \mathbb{R}^{N-1}$.

So for any $v_h \in \mathbb{R}^{N-1}$, we have $v_h = \sum_{m=1}^{N-1} c_{m,h} \Phi_{m,h}$.

Moreover, observe that $\Delta_h$ is symmetric and so $\Phi_{1,h}, \dots, \Phi_{N-1,h}$ are orthogonal in the standard scalar product:

\begin{equation*}
<v_h, w_h> = \sum_{x \in I_h} v_h(x) w_h(x)
\end{equation*}

of $\mathbb{R}^{N-1}$. The scalar product introduced in $L(I_h)$ is:

\begin{equation*}
<v_h, w_h>_h = h <v_h, w_h>
\end{equation*}

So $\Phi_{1,h}, \dots, \Phi_{N-1,h}$ are orthogonal also in the scalar product $<., .>_h$

\subsubsection{Exercise}

For $v_h \in L(I_h)$, prove that:

\begin{equation*}
c_{m,h} = \frac{<v_h, \Phi_{m,h}>}{||\Phi_{m,h}||_h^2} = <\Phi_{m,h}, \Phi_{m,h}> \quad, m \in \{1, \dots, N-1\}
\end{equation*}

Moreover, prove that the discrete Parceval's identity:

\begin{equation*}
||v_h||_h^2 = \sum_{m=1}^{N-1} c_{m,h}^2 ||\Phi_{m,h}||_h^2
\end{equation*}

We have:
\begin{equation*}
<v_h, \Phi_{m,h}>_h = <\sum_{n=1}^{N-1} c_{n,h} \Phi_{n,h}, \Phi_{m,h}>_h = \sum_{n=1}^{N-1} c_{n,h} \underbrace{<\Phi_{n,h}, \Phi_{m,h}>_h}_{=0 \;\; \forall n \neq m} = c_{m,h} <\Phi_{m,h}, \Phi_{m,h}>
\end{equation*}

\begin{equation*}
||v_h||_h^2 = <v_h, v_h>_h = <\sum_{n=1}^{N-1} c_{n,h} \Phi_{n,h}, \sum_{m=1}^{N-1} c_{m,h} \Phi_{m,h}>_h = \sum_{n=1}^{N-1} \sum_{n=1}^{N-1} c_{n,h} c_{m,h} <\Phi_{n,h}, \Phi_{m,h}>_h = \sum_{m=1}^{N-1} c_{m,h}^2 \underbrace{<\Phi_{m,h}, \Phi_{m,h}>}_{= ||\Phi_{m,h}||}
\end{equation*}

Now we're ready for the band of:

\begin{equation*}
||\Delta_h^{-1} || = \sup_{f_h \in L(I_h)} \frac{||v_h||_h}{||f_h||_h}
\end{equation*}

where $v_h$ is the solution of:

\begin{equation*}
\Delta_h v_h = f_h
\end{equation*}

Let:
\begin{equation*}
v_h = \sum_{m=1}^{N-1} c_{m,h} \Phi_{m,h}
\end{equation*}
the discrete Fourier series for $v_h$.

We have:
\begin{equation*}
f_h = \Delta_h v_h = \Delta_h \sum_{m=1}^{N-1} c_{m,h} \Phi_{m,h} = \sum_{m=1}^{N-1} c_{m,h} \underbrace{\Delta_h\Phi_{m,h}}_{= \lambda_{m, h}\Phi_{m,h}} = \sum_{m=1}^{N-1} (c_{m,h} \lambda_{m, h}) \Phi_{m,h}
\end{equation*}
which is the discrete Fourier series of $f_h$.

Now we use the discrete Parseval's identity:
\begin{equation*}
||f_h||_h^2 = \sum_{m=1}{N-1} (c_{m,h} \lambda_{m,h})^2 ||\Phi_{m,h}||_h^2 = \sum_{m=1}{N-1} c_{m,h}^2 \underbrace{\lambda_{m,h}^2}_{= |\lambda_{m,h}|^2} ||\Phi_{m,h}||_h^2
\end{equation*}

\begin{equation*}
-8 \ge \lambda_{1,h} \ge \lambda_{2,h} \ge \dots \implies 8 \le |\lambda_{1,h}| \le |\lambda_{2,h}| \le \dots 
\end{equation*}

\begin{equation*}
\ge \sum_{m=1}^{N-1} 8^2 c_{m,h}^2 ||\Phi_{m,h}||_h^2 = 8^2 \underbrace{\sum_{m=1}^{N-1} c_{m,h}^2 ||\Phi_{m,h}||_h^2}_{= ||v_h||_h^2}
\end{equation*}

So, we have proved:
\begin{equation*}
||v_h||^2 \le \frac{1}{8^2} \dots
\end{equation*}

Since $f_h \in L(I_h)$ is arbitrary we have:
\begin{equation*}
||\Delta_h^{-1}|| \sup_{f_h \in L(I_h)} \frac{||v_h||_h}{||f||_h} \le \frac{1}{8}
\end{equation*}

Now we pass to the 2D case.

In the 2D case:
\begin{equation*}
\Omega = (0, 1)^2 = I^2
\end{equation*}

\begin{equation*}
\Omega_h = I^2 = \{(mh, nh): m,n \in \{1, \dots, N-1 \}\}
\end{equation*}

$L(\Omega_h)$ is isomorphic to $\mathbb{R}^{(N-1)^2}$:
\begin{equation*}
L(\Omega_h) = \mathbb{R}^{(N-1)^2}
\end{equation*}

We introduce, for $m,n \in \{1, \dots, N-1\}$, the functions $\Phi_{m,n,h} \in L(\Omega_h)$ given by:
\begin{equation*}
\Phi_{m,n,h}(x,y) = \Phi_{m,h}(x) \Phi_{n,h}(y) \quad (x,y) \in \Omega_h
\end{equation*}

where $\Phi_{m,h}$ and $\Phi_{n,h}$ were defined in the 1D case.

The scalar product on $L(\Omega_h)$ is:
\begin{equation*}
<v_h, w_h> = h^2 \sum_{m=1}^{N-1} \sum_{n=1}^{N-1} v_h (m_h, n_h) w_h(m_h, n_h)
\end{equation*}

and the $L^2$ norm is:
\begin{equation*}
||v_h||_h = \sqrt{<v_h, v_h>} = \sqrt{h^2 \sum_{m=1}^{N-1} \sum_{n=1}^{N-1} v_h(m_h, n_h)^2}
\end{equation*}
for $v_h, w_h \in L(\Omega_h)$.

For $m, n \in \{1, \dots, N-1\}$, $\Phi_{m,n,h}$ is an eigenvector of the 2D discrete laplacian $\Delta_h$.

\begin{equation*}
\Delta_h \Phi_{m,n,h}(x,y) = \frac{\Phi_{m,n,h}(x-h, y) - 2 \Phi_{m,n,h}(x, y) + \Phi_{m,n,h}(x+h, y)}{h^2} + \frac{\Phi_{m,n,h}(x, y-h) - 2 \Phi_{m,n,h}(x, y) + \Phi_{m,n,h}(x, y+h)}{h^2}
\end{equation*}

\begin{equation*}
= \frac{\Phi_{m,h}(x - h) \Phi_{n,h}(y) - 2 \Phi_{m,h}(x) \Phi_{n,h}(y) + \Phi_{m,h}(x+h)\Phi_{n,h}(y)}{h^2} + \frac{\Phi_{m,h}(x) \Phi_{n,h}(y - h) - 2 \Phi_{m,h}(x) \Phi_{n,h}(y) + \Phi_{m,h}(x)\Phi_{n,h}(y+h)}{h^2}
\end{equation*}

\begin{equation*}
= \frac{\Phi_{m,h}(x - h) - 2 \Phi_{m,h}(x) + \Phi_{m,h}(x+h) }{h^2} \Phi_{m,h}(y) +  \Phi_{m,h}(x) \frac{\Phi_{n,h}(y - h) - 2\Phi_{n,h}(y) + \Phi_{n,h}(y+h)}{h^2}
\end{equation*}

\begin{equation*}
= \lambda_{m,h} \Phi_{m,h}(x) \Phi_{n,h}(y) + \Phi_{m,h}(x) \lambda_{n,h} \Phi_{n,y}(y) = (\lambda_{m,h} + \lambda_{n,h}) \underbrace{\Phi_{m,h}(x) \Phi_{n,h}(y)}_{= \Phi_{m,n,h}(x,y)}
\end{equation*}

We have proved that $\Phi_{m,n,h}$ is an eigenvector of 2D $\Delta_h$ with relevant eigenvalue $\lambda_{m, h} + \lambda_{n,h}$.

Now, we prove that $\Phi_{m,n,h}, m,n \in \{1, \dots, N-1\}$ are orthogonal in the scalar product of $L(\Omega_h)$ we have introduced.

For $m,n,p,q \in \{1, \dots, N-1\}$ with $(m, n) \neq (p, q)$:
\begin{equation*}
<\Phi_{m,n,h}, \Phi_{p,q,h}>_h = h^2 \sum_{k=1}^{N-1} \sum_{l=1}^{N-1} \Phi_{m,n,h}(kh, lh) \Phi_{p,q,h}(kh, lh)
\end{equation*}

\begin{equation*}
= h^2 \sum_{k=1}^{N-1} \sum_{l=1}^{N-1} \Phi_{m,h}(kh)\Phi_{n,h}(lh) \Phi_{p,h}(kh)\Phi_{q,h}(lh)
\end{equation*}

\begin{equation*}
= h^2 (\sum_{k=1}^{N-1} \Phi_{m,h}(kh)\Phi_{p,h}(kh)) (\sum_{l=1}^{N-1} \Phi_{n,h}(lh)\Phi_{q,h}(lh))
\end{equation*}

\begin{equation*}
= \underbrace{<\Phi_{m,h}, \Phi_{p,h}>_h}_{1D scalar product} \cdot \underbrace{<\Phi_{n,h}, \Phi_{q,h}>_h}_{1D scalar product}
\end{equation*}

Remember $(m,n) \neq (p, q)$:
\begin{equation*}
m \neq p \text{ or } m n \neq q
\end{equation*}
so the above scalar product is zero.

Since the $(N-1)^2$ functions $\Phi_{m,n,h}$, $m,n \in \{1, \dots, N-1\}$ are nonzero orthogonal functions, they $(N-1)^2$ linearly independent functions of $\mathbb{R}^{(N-1)^2}$ and so they constitute a bases for $\mathbb{R}^{(N-1)^2}$.

So any $v_h \in L(\Omega_h)$ can be written as:
\begin{equation*}
v_h = \sum_{m=1}^{N-1} \sum_{n=1}^{N-1} \dots
\end{equation*}

Exactly as in the 1D case, we have:
\begin{equation*}
c_{m,n,h} = \frac{<v_h, \Phi_{m,n,h}>_h}{||\Phi_{m,n,h}||} \dots
\end{equation*}

\dots

Now we are ready to give a bond for 
\begin{equation*}
||\Delta_h^{-1}|| = \sup_{f_h \in L(\Omega_h)} \frac{||v_h||_h}{||f_h||_h}
\end{equation*}

where $v_h$ is the solution of: $\Delta_h v_h = f_h$

\dots

Let $f_h \in L(\Omega_h)$ and let $v_h$ be the solution of $\Delta_h v_h = f_h$. Let:
\begin{equation*}
v_h = \sum_{m=1}^{N-1} \sum_{n=1}^{N-1} c_{m,n,h} \Phi_{m,n,h}
\end{equation*}
be the Fourier series of $v_h$.

We have:
\begin{equation*}
f_h = \Delta_h v_h = \sum_{m=1}^{N-1} \sum_{n=1}^{N-1} c_{m,n,h} \underbrace{\Delta_h \Phi_{m,n,h}}_{(\lambda_{m,h} + \lambda_{n,h})\Phi_{m,n,h}} = \sum_{m=1}^{N-1} \sum_{n=1}^{N-1} c_{m,n,h} (\lambda_{m,h} + \lambda_{n,h}) c_{m,n,h} \Phi_{m,n,h}
\end{equation*}
which is the discrete Fourier series of $f_h$.

So:
\begin{equation*}
||f_h||_h^2 = \sum_{m=1}^{N-1} \sum_{n=1}^{N-1} \underbrace{(\lambda_{m,h} + \lambda_{n,h})^2}_{||\Phi_{m,n,h}||_h^2} c_{m,n,h}^2
\end{equation*}

\begin{equation*}
= \sum_{m=1}^{N-1} \sum_{n=1}^{N-1} (\underbrace{|\lambda_{m,h}|}_{\ge 8} + \underbrace{|\lambda_{n,h}|}_{\ge 8})^2 c_{m,n,h}^2 \ge \sum_{m=1}^{N-1} \sum_{n=1}^{N-1} 16^2 c_{m,n,h}^2
\end{equation*}

Then we have:
\begin{equation*}
||v_h||_h^2 \le \frac{1}{16^2} ||f_h||_h^2
\end{equation*}

\begin{equation*}
||v_h||_h \le \frac{1}{16} ||f_h||_h
\end{equation*}

Since $f_h \in L(\Omega_h)$ is arbitrary:
\begin{equation*}
||\Delta_h^{-1}|| = \sup_{f_h \in L(\Omega_h)} \frac{||v_h||_h}{||f_h||_h} \le \frac{1}{16}
\end{equation*}

Remember we have:
\begin{equation*}
\Delta_h \underbrace{e_h}_{convergence error} = - \underbrace{\epsilon_h}_{consistency error}
\end{equation*}

So:
\begin{equation*}
||e_h||_h \le ||\Delta_h^{-1}|| ||\epsilon_h||_h \le \frac{1}{16} ||\epsilon_h||_h
\end{equation*}

In case of the $L^\infty$ norm, we had:
\begin{equation*}
||e_h||_{L^\infty(\Omega_h)} \le \frac{1}{8} ||\epsilon_h||_{L^\infty(\Omega_h)}
\end{equation*}


\subsubsection{Exercise}

Prove that, for $v_h \in L(\Omega_h)$, we have:
\begin{equation*}
||v_h|| \le ||v_h||_{L^\infty(\Omega_h)}
\end{equation*}

and then conclude with the estimate:
\begin{equation*}
||e_h||_h = O(h^2), h \rightarrow 0
\end{equation*}

\begin{equation*}
||v_h||_h = \sqrt{h^2 \sum_{m=1}^{N-1} \sum_{n=1}^{N-1} \underbrace{v_h(kh, nh)^2}_{\le ||v_h||_{L^\infty(\Omega_h)}^2}}
\end{equation*}

\begin{equation*}
\le \sqrt{h^2 \sum_{m=1}^{N-1} \sum_{n=1}^{N-1} ||v_h||_{L^\infty(\Omega_h} } = ||v_h||_{L^\infty(\Omega_h} \sqrt{h^2 \sum_{m=1}^{N-1} \sum_{n=1}^{N-1} 1}
\end{equation*}

\begin{equation*}
= ||v_h||_{L^\infty(\Omega_h} \underbrace{\sqrt{h^2 (N-1)}}_{= h (N-1)}
\end{equation*}

\begin{equation*}
\le ||v_h||_{L^\infty(\Omega_h} \underbrace{h N}_{= 1} = ||v_h||_{L^\infty(\Omega_h}
\end{equation*}

Then:
\begin{equation*}
||e_h||_h \le \frac{1}{16} ||\epsilon_h||_h \le \frac{1}{16} \overbrace{||\epsilon_h||_{L^\infty(\Omega_h)}}^{= O(h^2), \, h \rightarrow 0}
\end{equation*}

and then:
\begin{equation*}
\qquad\qquad\qquad\qquad\qquad\qquad ||e_h||_h = O(h^2), \, h \rightarrow 0   \qquad\qquad\qquad\qquad\qquad\qquad \blacksquare
\end{equation*}

\section{Some notes needs to be copied from paper here}

...

We proved: for $v: \overline{\Omega} \rightarrow \mathbb{R}$ of class $C^3$:

..

The order of consistency error is only $O(h)$, not $O(h^2)$ as in the case of the square.
We have the discrete problem:

\begin{equation*}
\Delta_h u_h (x, y) = f(x, y), \quad (x, y) \in \Omega_h
\end{equation*}
\begin{equation*}
u_h(x, y) = g(x, y), \quad (x,y) \in \Gamma_h
\end{equation*}

This is a linear system of M unknowns $u_h(x, y), \quad (x,y) \in \Omega_h$, where $M$ is the number of points in $\Omega_h$, into $M$ equations; we have an equation for any point in $\Omega_h$.

With respect to the case of the square, the matrix of the system is in general non-symmetric, because:

\begin{equation*}
\Delta_h u_h(x, y) = \frac{2}{h_1 (h_1 + h_2)} u_h(x - h_1, y) - \frac{2}{h_1 h_2} (x, y) + \frac{2}{h_2 (h_1 + h_2)} (x + h_2, y) + \frac{2}{h_3 (h_3 + h_4)} u_h(x, y - h_3) - \frac{2}{h_3 h_4} (x, y) + \frac{2}{h_3 (h_3 + h_4)} (x, y + h_4) = f(x, y), \quad (x,y) \in \Omega_h
\end{equation*}

Symmetric matrix: $a_{ij} = a_{ji}$
the coefficient of the \textit{j}-th matrix, in the \textit{i}-th equation is equal to the coefficient of the \textit{i}-th matrix in the \textit{j}-th equation.
The coefficient in the \textit{j}-th equation of the \textit{i}-th unknown is 

\begin{equation*}
\frac{2}{h_2 (h_2 + \hat{h_2})} u_h(x, y) - \frac{2}{h_2 \hat{h_2}} u_h(x + h_2, y)+ \frac{2}{\hat{h_2}(h_2 + \hat{h_2})} u_h(x + \hat{h_2}, y)
\end{equation*}

Characteristics of the matrix that are maintained in the general case:
\begin{itemize}
	\item sparsity (the matrix is sparse): in any row there are at most 5 non-zero elements;
	\item diagonal elements are negative and off-diagonal elements are positive
	\item diagonal dominance
\end{itemize}

An MxM matrix A is called diagonal dominant if:

\begin{equation*}
\forall i \in \{1, \dots, M\}: |a_{ii}| \ge \sum_{j=1, j \neq i}^{M} |a_{ij}|
\end{equation*}

Maximum sum of the off-diagonal elements is:

\begin{equation*}
\frac{2}{h_1 (h_1 + h_2)} + \frac{2}{h_2 (h_1 + h_2)} + \frac{2}{h_3 (h_3 + h_4)} + \frac{2}{h_4 (h_3 + h_4)} = \frac{2}{h_1 + h_2} (\frac{1}{h_1} + \frac{1}{h_2}) + \frac{2}{h_3 + h_4 (\frac{1}{h_3} + \frac{1}{h_4})}
\end{equation*}

\begin{equation*}
\frac{2}{h_1 + h_2} \frac{h_1 + h_2}{h_1 h_2} + \frac{2}{h_3 + h_4} \frac{h_3 + h_4}{h_3 h_4} = \frac{2}{h_1 + h_2} + \frac{2}{h_3 + h_4}
\end{equation*}

We have, by stability plus consistency of order one, convergence of order one. But
\begin{enumerate}
	\item Points in $\mathring{\Omega}_h$ have consistency error $O(h^2)$;
	\item number of points in $\Omega_h \mathring{\Omega}_h$ (where the consistency error is $O(h)$)  divided by number of points in $\mathring{\Omega}_h$ (where the consistency error is $O(h^2)$) $= O(h)$
	\item Points of $\Omega_h \\ \mathring{\Omega}_h$ are at a distance $O(h)$ of the boundary $\Gamma_h$, where the solution is known exactly.
\end{enumerate}

